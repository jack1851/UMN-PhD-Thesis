\documentclass[aspectratio=169]{beamer}

\input{../preamble}

%--------------------------------------------------
% Geometry and Scale Settings
% Adjust the scaling factor for physical dimensions.
%--------------------------------------------------
\geometry{mag=1600, truedimen}

%--------------------------------------------------
% University of Minnesota Color Definitions
%--------------------------------------------------
\definecolor{UMNMaroon}{RGB}{122,0,25}
\definecolor{UMNLightGold}{RGB}{255,215,95}
\definecolor{UMNGold}{RGB}{255,204,51}
\definecolor{UMNStormy}{RGB}{64,77,91}
\definecolor{UMNSunny}{RGB}{0,149,182}
\definecolor{UMNLightGray}{RGB}{255,222,122}

%--------------------------------------------------
% Package Imports
% Load additional packages required for the presentation.
%--------------------------------------------------
\usepackage{xspace}         % Ensures \xspace works properly with custom macros.
\usepackage{textpos}        % Provides absolute positioning.
\usepackage{epstopdf}       % Converts EPS images to PDF.
\usepackage{graphicx}       % Handles image inclusion.
\usepackage{beamerthemesplit} % Legacy beamer theme; consider using a modern theme if desired.
\usepackage{soul}           % Allows text highlighting and underlining.
\usepackage{xifthen}        % Enables conditional statements.
\usepackage{listings}       % For code formatting.
\usepackage{hyperref}       % For hyperlink support in the generated PDF.
\usepackage{bbding}         % Provides additional symbols (e.g., checkmarks).

%--------------------------------------------------
% Hyperref Configuration
% Customize link colors and PDF metadata.
%--------------------------------------------------
\hypersetup{
    colorlinks=true,
    linkcolor=UMNMaroon,
    urlcolor=UMNStormy,
    citecolor=UMNSunny,
    pdftitle={Presentation Title},
    pdfauthor={Your Name}
}

%--------------------------------------------------
% Listings Settings for Code Snippets
%--------------------------------------------------
\definecolor{backcolour}{rgb}{0.95,0.95,0.92}
\lstdefinestyle{mystyle}{
    backgroundcolor=\color{UMNLightGray},
    commentstyle=\color{UMNStormy},
    keywordstyle=\color{UMNMaroon},
    numberstyle=\tiny\color{UMNMaroon},
    stringstyle=\color{UMNSunny},
    basicstyle=\ttfamily\footnotesize,
    emph={ldmx },
    emphstyle=\color{UMNMaroon},
    breakatwhitespace=false,
    breaklines=true,
    captionpos=b,
    keepspaces=true,
    numbers=left,
    numbersep=5pt,
    showspaces=false,
    showstringspaces=false,
    showtabs=false,
    tabsize=2
}
\lstset{style=mystyle}

%--------------------------------------------------
% Custom Commands and Environments
%--------------------------------------------------

%--- Code Listing Commands ---%
% Import code from an external file.
% #1: Options for the formatter.
% #2: File path to the code file.
\newcommand{\codefile}[2]{%
    \lstinputlisting[#1]{#2}
}%

% Inline code command with a styled box.
% #1: Inline code text.
\newcommand{\code}[1]{%
  \tikz[baseline=(codebox.base)]{
    \node[inner sep=1pt,outer sep=0pt,draw=UMNStormy,line width=0.05mm,fill=UMNLightGray,rounded corners=0.03cm] (codebox) 
      {\textcolor{UMNSunny}{#1}};
  }%
}%

% Styled title command within a frame for subsection headings.
% #1: Text to style.
\newcommand{\framesection}[1]{%
  \boldcol{UMNStormy}{#1}\hspace{1em}%
}%

%--- List Item Symbols ---%
% Define checkmark and crossmark symbols with custom colors.
\def\checkmark{{\color{UMNSunny}\Checkmark}}
\def\crossmark{{\color{UMNMaroon}\XSolidBrush}}

% Define "done" and "todo" list items.
\def\done{\item[$\color{UMNMaroon}\boxtimes$]}
\def\todo{\item[$\color{UMNMaroon}\square$]}

% Define "good" and "bad" list items.
\def\good{\item[\checkmark]}
\def\bad{\item[\crossmark]}

%--- Hyperlink Button ---%
% Creates a clickable button linking to a URL.
% #1: URL to link to.
% #2: Button text.
\newcommand{\hlink}[2]{%
    \href{#1}{\beamergotobutton{#2}}
}%

%--- Bold and Colored Text ---%
% Boldens and colors text.
% #1: Color.
% #2: Text.
\newcommand{\boldcol}[2]{%
    {\color{#1}\textbf{#2}}\xspace
}%

%--- One Plot Slide ---%
% Creates a slide with one plot that fills most of the frame.
% #1: Slide Title.
% #2: Slide Subtitle.
% #3: File path to the image.
\newcommand{\oneplotslide}[3]{%
    \begin{frame}
        \frametitle{#1}
        \framesubtitle{#2}
        \begin{figure}
            \centering
            \includegraphics[width=\textwidth,height=0.8\textheight,keepaspectratio]{#3}
        \end{figure}
     \end{frame}
}%

%--- Four Tiles Slide ---%
% Arranges content into four tiles on a slide.
% #1: Slide Title.
% #2: Slide Subtitle.
% #3: Top left content.
% #4: Top right content.
% #5: Bottom left content.
% #6: Bottom right content.
\newcommand{\fourtileslide}[6]{%
    \begin{frame}
        \frametitle{#1}
        \framesubtitle{#2}
        \begin{figure}[h]
            \centering
            \begin{tabular}{cc}
                #3 & #4 \\
                #5 & #6
            \end{tabular}
        \end{figure}
    \end{frame}
}%

%--- Four Plots Slide ---%
% A slide that tiles four plots; empty paths yield a blank tile.
% #1: Slide Title.
% #2: Slide Subtitle.
% #3: Top left image file path.
% #4: Top right image file path.
% #5: Bottom left image file path.
% #6: Bottom right image file path.
\newcommand{\fourplotslide}[6]{%
    \fourtileslide{#1}{#2}%
        {\ifthenelse{\isempty{#3}}{}{\includegraphics[height=0.4\textheight]{{#3}}}}%
        {\ifthenelse{\isempty{#4}}{}{\includegraphics[height=0.4\textheight]{{#4}}}}%
        {\ifthenelse{\isempty{#5}}{}{\includegraphics[height=0.4\textheight]{{#5}}}}%
        {\ifthenelse{\isempty{#6}}{}{\includegraphics[height=0.4\textheight]{{#6}}}}%
}%

%--- Figure with Comments ---%
% Creates a slide with an image on the left and commentary on the right.
% #1: File path of the image.
\newenvironment{figurecomments}[1]{%
  \begin{columns}
    \begin{column}{0.6\textwidth}
      \begin{figure}
        \centering
        \includegraphics[height=0.7\textheight]{#1}
      \end{figure}
    \end{column}
    \begin{column}{0.4\textwidth}
      \footnotesize
}{%
    \end{column}
  \end{columns}
}%

%--- Section Title Slide ---%
% Creates a full-slide title card for sections.
% #1: Section title.
\newcommand{\sectionframe}[1]{%
\begin{frame}
    \vfill
    \centering
    \begin{beamercolorbox}[sep=8pt,center,shadow=true,rounded=true]{title}
        \usebeamerfont{title}#1\par%
    \end{beamercolorbox}
    \vfill
\end{frame}
}%

%--- Backup Slides Environment ---%
% Wrap backup slides to exclude them from the main slide count.
% Optionally, you can title the backup section (default is "Questions").
\newenvironment{backup}[1][Questions]{%
  \sectionframe{#1}
  \newcounter{finalframe}
  \setcounter{finalframe}{\value{framenumber}}
}{%
  \setcounter{framenumber}{\value{finalframe}}
}%

\usepackage{pgfpages}
\setbeamertemplate{note page}[plain]
\setbeameroption{show notes}
\addtobeamertemplate{note page}{}{\thispdfpagelabel{notes:\insertframenumber}}

\newcommand{\ssection}[1]{%
  \section{#1}%
  \sectionframe{#1}%
}%

\newenvironment{sframe}[1]{%
  \subsection{#1}%
  \begin{frame}{#1}%
}{%
  \end{frame}%
}%

\title[$\mathrm{W_R}$ Boson and Heavy Neutrino Search]{%
  Search for a Right-Handed W Boson and Heavy Neutrino of the Left-Right Symmetric Standard Model%
}

\begin{document}

\begin{frame}
  \maketitle
\end{frame}

\note[itemize]{
\item Hello everyone, for those of you who don't know me, my name is Billy.
  I am one of Jeremy's graduate students, and in this seminar I am going to
  talk about our search for a right-handed W boson and heavy neutrino of the 
  Left-Right Symmetric Standard Model. 
}

\begin{frame}{Outline}
  \begin{columns}
    \begin{column}{0.5\textwidth}
      \tableofcontents
    \end{column}
    \begin{column}{0.5\textwidth}
      \includegraphics[width=0.9\textwidth]{../figures/experiment/cms-slice}
      \\
      \resizebox{0.9\textwidth}{!}{\begin{tikzpicture}
    \begin{feynman}
      %
      % -- Define the incoming quarks on the left
      \vertex (qbar) at (-2, 1.5) {\(\bar{q}'\)};
      \vertex (q)    at (-2,-1.5) {\(q\)};
      
      % -- Central vertex where q qbar' -> W_R
      \vertex [dot] (wr) at (0,0) {};
      
      % -- First decay: W_R -> N_\ell + l
      \vertex [dot] (Nl) at (2.5,0) {};
      \vertex (l1)  at (4,-1.5) {\(\ell^{+}\)};
      \vertex (N) at (4, 1.5) {};
      
      % -- Second decay: N_\ell -> l + W_R
      %    Then W_R -> 2 jets
      \vertex [dot] (wrl) at (4, 1.5) {};
      \vertex (l2) at (5.5,  3) {\(\ell^{-}\)};
      \vertex (wr2) at (6,  1) {};

      %    Then W_R -> 2 jets
      \vertex [dot] (jj) at (6, 1) {};
      \vertex (jet1) at (7.5,  1.7) {jet};
      \vertex (jet2) at (7.5, 0.3) {jet};
      
      % -- Draw the diagram
      \diagram*{
        % Incoming quarks
        (qbar) -- [fermion] (wr) -- [fermion] (q),
        
        % First decay of W_R
        (wr) -- [boson, edge label=\(W_R\)] (Nl),
        (l1) -- [fermion] (Nl),
        (Nl) -- [fermion, edge label=\(N_\ell\)] (wrl),
        
        % Decay of heavy neutrino N_l
        (wrl) -- [fermion] (l2),
        (wrl) -- [boson, edge label=\(W_R^{*}\)] (wr2),

        % Decay of second W_R into jets
        (jet1) -- [fermion] (jj) -- [fermion] (jet2),
      };
    \end{feynman}
  \end{tikzpicture}}
    \end{column}
  \end{columns}
\end{frame}

\section{Background}

\subsection{Parity Violation and Handedness}

\begin{frame}{Parity Violation in $\beta$ Decay}
  \begin{itemize}
    \item The parity operator $\hat{P}$ corresponds to a discrete 
      transformation $x \rightarrow -x$, etc.
    \item In 1957 C.S. Wu studied the beta decay:
      $\ce{^{60}_{27}Co \rightarrow ^{60}_{28}Ni + e^{-} + \bar{\nu_{e}}}$
  \end{itemize}
  \begin{columns}
    \begin{column}{0.65\textwidth}
      \begin{figure}
        \centering
        \includegraphics[width=\textwidth]{wu-experiment.png}
      \end{figure}
    \end{column}
    \begin{column}{0.3\textwidth}
      \begin{block}{}
        If parity were conserved: expect equal rate for producing $e^{-}$ in directions
            along and opposite to the nuclear spin.
      \end{block}
    \end{column}
  \end{columns}
  \begin{itemize}
    \item Observed \boldcol{UMNMaroon}{electrons emitted preferentially} in direction 
      opposite to applied field.
  \end{itemize}
  \begin{block}{}
    \centering
      Conclude \boldcol{UMNSunny}{parity is violated} in weak interactions.
  \end{block}
\end{frame}

\begin{frame}{Helicity States}
  \begin{block}{Definition}
    Define the \boldcol{UMNSunny}{Helicity} of a particle as the projection of its spin onto the direction of momentum.
  \end{block}
  \begin{columns}
    \begin{column}{0.60\textwidth}
      \begin{figure}
        \centering
        \input{../figures/theory/helicity.tex}
        \caption{Illustration of helicity states in muon decay. 
          $e^{+}$ and $\bar{\nu}_{\mu}$ are emitted as \boldcol{UMNMaroon}{right-handed} helicity states. 
          $\nu_e$ is emitted as a \boldcol{UMNMaroon}{left-handed} helicity state.}
      \end{figure}
    \end{column}
    \begin{column}{0.40\textwidth}
      \begin{block}{A Good Quantum Number}
        \centering
        $[\hat{H}_D, \hat{S} \cdot \hat{p}] = 0$
        \begin{itemize}
          \item Possible to find solutions of the Dirac equation which are also 
            eigenstates of Helicity.
        \end{itemize}
      \end{block}
    \end{column}
  \end{columns}
  \centering
  \boldcol{UMNSunny}{But it is important to remember that helicity is not Lorentz invariant.}
\end{frame}

\begin{frame}{Chiral States}
  \begin{itemize}
    \item \boldcol{UMNSunny}{Chirality}: The Lorentz invariant ``version'' of helicity and an intrinsic property of particles.
      \begin{itemize}
        \item In general, define the eigenstates of $\gamma^{5}$ as \boldcol{UMNMaroon}{left} and \boldcol{UMNMaroon}{right-handed chiral} states
      \end{itemize}
    \item \boldcol{UMNSunny}{Ultra-relativistic limit}:
      $E >> m$
      \begin{itemize}
        \item Chiral eigenstates correspond to helicity eigenstates.
      \end{itemize}
    \item \boldcol{UMNSunny}{Projection Operators}
      \begin{itemize}
        \item Project out the chiral eigenstates:
      \end{itemize}
      $$
        \boxed{P_{L} = \frac{1}{2}\left(1 - \gamma^{5}\right) \quad P_{R} = \frac{1}{2}\left(1 + \gamma^{5}\right)}
      $$
    \item \boldcol{UMNSunny}{Spinor Decomposition}
      \begin{itemize}
        \item Can write any spinor in terms of its left and right-handed chiral components
      \end{itemize}
      $$
        \Psi = \Psi_{L} + \Psi_{R} = P_{L}\Psi + P_{R}\Psi = \frac{1}{2}\left(1 - \gamma^{5}\right)\Psi + \frac{1}{2}\left(1 + \gamma^{5}\right)\Psi
      $$
  \end{itemize}
\end{frame}

\subsection{Left-Right Symmetric Extensions}

\begin{frame}{Charged Weak Interactions}
  \begin{columns}
    \begin{column}{0.5\textwidth}
      \begin{figure}
        \centering
        \input{../figures/theory/left-w-vertex}
        \caption{The SM left-handed charged weak vertex.}
      \end{figure}
      \begin{itemize}
        \item Only \boldcol{UMNMaroon}{left-handed chiral} components of 
          particle spinors participate in charged weak interactions.
        \item No theoretical explanation as to \emph{why} 
          the weak force is left-handed.
      \end{itemize}
    \end{column}
    \begin{column}{0.5\textwidth}
      \begin{figure}
        \centering
        \input{../figures/theory/right-w-vertex}
        \caption{The LRSM right-handed weak vertex.}
      \end{figure}
      \begin{itemize}
        \item Introduce $W_{R}^{\pm}$ bosons to couple to \boldcol{UMNMaroon}{right-handed chiral} components of 
        particle spinors.
        \item Under parity, left and right-handed vertices transform into each other.
      \end{itemize}
    \end{column}
  \end{columns}
\end{frame}

\begin{frame}{Left-Right Symmetric Models}
  \begin{block}{Postulate of Left-Right Symmetric Models}
    The weak force's left-handed nature is a low energy phenomenon -- 
    the result of a broken left-right symmetry that is restored at a multi-TeV energy scale.
  \end{block}
  \vfill
  \begin{table}
    \begin{tabular}{lcc}
    \toprule
    & \textbf{Electroweak Standard Model} 
    & \textbf{Left-Right Symmetric Model} \\ 
    \midrule
    \textbf{Gauge Group} 
      & \(\mathrm{SU}(2)_{L} \times \mathrm{U}(1)\) 
      & \(\textcolor{red}{\mathrm{SU}(2)_{R}} \times \mathrm{SU}(2)_{L} \times \mathrm{U}(1)\) \\[6pt]
    \textbf{Fermions} 
      & \(Q_{L} = (u_i,\,d_i)_{L},\quad L_{L} = (\ell_i,\,\nu_i)_{L}\) 
      & \(Q_{R} = (u_i,\,d_i)_{R},\quad L_{R} = (\ell_i,\,n_i)_{R}\) \\[6pt]
    \textbf{Gauge Bosons} 
      & \(W_{L}^{\pm},\ Z,\ \gamma\) 
      & \(W_{R}^{\pm},\ Z,\ Z'\) \\ 
    \bottomrule
  \end{tabular}
    \caption{Summary of the Left-Right Symmetric SM
    with new extensions in \textcolor{red}{red}.}
  \end{table}
\end{frame}

\begin{frame}{Seesaw Mechanism}
  \begin{figure}
    \centering
    \includegraphics[width=0.70\textwidth]{seesaw-mechanism.png}
    \caption{In LRSM models the lightness of SM neutrinos is explained
    via the seesaw mechanism. Artwork by Sandbox Studio, Chicago with Ana Kova.}
  \end{figure}
\end{frame}

\begin{frame}{Feynman Diagram}
  \begin{figure}
    \centering
    \begin{tikzpicture}
    \begin{feynman}
      %
      % -- Define the incoming quarks on the left
      \vertex (qbar) at (-2, 1.5) {\(\bar{q}'\)};
      \vertex (q)    at (-2,-1.5) {\(q\)};
      
      % -- Central vertex where q qbar' -> W_R
      \vertex [dot] (wr) at (0,0) {};
      
      % -- First decay: W_R -> N_\ell + l
      \vertex [dot] (Nl) at (2.5,0) {};
      \vertex (l1)  at (4,-1.5) {\(\ell^{+}\)};
      \vertex (N) at (4, 1.5) {};
      
      % -- Second decay: N_\ell -> l + W_R
      %    Then W_R -> 2 jets
      \vertex [dot] (wrl) at (4, 1.5) {};
      \vertex (l2) at (5.5,  3) {\(\ell^{-}\)};
      \vertex (wr2) at (6,  1) {};

      %    Then W_R -> 2 jets
      \vertex [dot] (jj) at (6, 1) {};
      \vertex (jet1) at (7.5,  1.7) {jet};
      \vertex (jet2) at (7.5, 0.3) {jet};
      
      % -- Draw the diagram
      \diagram*{
        % Incoming quarks
        (qbar) -- [fermion] (wr) -- [fermion] (q),
        
        % First decay of W_R
        (wr) -- [boson, edge label=\(W_R\)] (Nl),
        (l1) -- [fermion] (Nl),
        (Nl) -- [fermion, edge label=\(N_\ell\)] (wrl),
        
        % Decay of heavy neutrino N_l
        (wrl) -- [fermion] (l2),
        (wrl) -- [boson, edge label=\(W_R^{*}\)] (wr2),

        % Decay of second W_R into jets
        (jet1) -- [fermion] (jj) -- [fermion] (jet2),
      };
    \end{feynman}
  \end{tikzpicture}
    \caption{$\mathrm{W_R}$ decay chain. 
    The final state observables are two same-flavor leptons
    and two jets.}
  \end{figure}
\end{frame}

\note[itemize]{
  \item This feynmann diagram gives us a sense of what exactly we are looking for. 
  \item We have two qaurks coming in, annihilating and producing a right-handed W. 
  That right-handed W decays to a lepton and a heavy neutrino. That heavy neutrino 
  then decays back into a same flavor lepton and an off-shell right handed W, which 
  then decays into two quarks which hadronize to jets. 
  \item So, the observable objects in the decay are two same flavor leptons, and two jets. 
}

\section{LHC and CMS}

\begin{frame}{LHC}
  \begin{figure}
    \centering
    \includegraphics[width=0.7\textwidth]{../figures/experiment/lhc-schematic.png}
    \caption{Schematic showing the overall layout of the LHC. Image credit -- CERN.}
  \end{figure}
\end{frame}

\begin{frame}{CMS Detector}
  \begin{figure}
    \centering
    \includegraphics[width=0.85\textwidth]{../figures/experiment/cms-slice.png}
    \caption{CMS detector (slice view). Image credit -- CERN.}
  \end{figure}
\end{frame}

\section{Search Strategy} % \ssection{Search Strategy}

\subsection{Resonance Searches}

\begin{frame}{Resonance Search}
  \begin{columns}
    \begin{column}{0.4\textwidth}
      \vfill
      {The cross section a \boldcol{UMNSunny}{Breit-Wigner distribution}:
      $$
        \sigma \propto \frac{1}{\left(m_{lljj}^{2} - m_{W_R}^{2}\right)^{2} + \Gamma^{2} m^{2}_{W_R}}
      $$}
      \begin{figure}
        \centering
        \input{../figures/theory/resonance}
      \end{figure}
    \end{column}
    \begin{column}{0.59\textwidth}
      \begin{block}{Bump Huntin'}
        Search for an excess of events in the \boldcol{UMNMaroon}{four object invariant mass} 
        spectrum consisting of the two leptons and two jets ($m_{\ell \ell j j}$).
      \end{block}
      \begin{figure}
        \centering
        \resizebox{0.9\linewidth}{!}{%
          \input{../figures/theory/wr-decay-annotated}%
      }
      \end{figure}
    \end{column}
  \end{columns}
\end{frame}

\begin{frame}[t]{Analysis Topologies}
  \begin{columns}[t]
    \begin{column}{0.32\textwidth}
      \centering
      \framesection{CMS Jets}
      \resizebox{0.99\linewidth}{!}{%
      % Author: Izaak Neutelings (May 2021)
% Description: hadronic top quark jet
\usetikzlibrary{calc,math,decorations.pathreplacing}

% TikZ arrow head
\tikzset{>=latex}

% COLORS
\colorlet{myblue}{blue!70!black}
\colorlet{mydarkblue}{blue!40!black}
\colorlet{mygreen}{green!40!black}
\colorlet{myred}{red!65!black}

% Cone styles
\tikzstyle{cone}     =[thin,blue!50!black,fill=blue!50!black!30]
\tikzstyle{conebase} =[cone,fill=blue!50!black!50]

% Macro to draw a jet cone from point #2 back to #1
% #1 (optional): color
% #2: apex coordinate name
% #3: base coordinate name
% #4: full opening angle (in degrees)
% #5: eccentricity factor
\newcommand\jetcone[5][blue]{{
  \pgfmathanglebetweenpoints%
    {\pgfpointanchor{#2}{center}}%
    {\pgfpointanchor{#3}{center}}%
  \edef\ang{#4/2}       % half‑opening angle
  \edef\e{#5}           % eccentricity
  \edef\vang{\pgfmathresult} % direction of axis
  \tikzmath{%
    coordinate \conevec;
    \conevec = (#2)-(#3);
    \x    = veclen(\conevecx,\conevecy)*\e*sin(\ang)^2;
    \y    = tan(\ang)*(veclen(\conevecx,\conevecy)-\x);
    \a    = veclen(\conevecx,\conevecy)*sqrt(\e)*sin(\ang);
    \b    = veclen(\conevecx,\conevecy)*tan(\ang)*sqrt(1-\e*sin(\ang)^2);
    \angb = acos(sqrt(\e)*sin(\ang));
  }
  \coordinate (tmpL) at 
    ($(#3)-(\vang:\x pt)+(\vang+90:\y pt)$);
  % Draw full ellipse for cone base
  \draw[thin,#1!40!black,rotate=\vang,
        top color=#1!50!black!80,
        bottom color=#1!40!black!80,
        shading angle=\vang]
    (#3) ellipse({\a pt} and {\b pt});
  % Draw the “cone side” as an arc + lines back to apex
  \draw[thin,#1!40!black,rotate=\vang,rounded corners=0.001pt,
        top color=#1!90!black!20,
        bottom color=#1!50!black!50,
        shading angle=\vang]
    (tmpL) arc(180-\angb:180+\angb:{\a pt} and {\b pt})
    -- (#2) -- cycle;
}}

\begin{tikzpicture}[scale=7]
  % Core points
  \coordinate (O)  at (0,0);
  \coordinate (BJ) at (56:1.1);  % b‑jet axis
  \coordinate (J1) at (12:1.0);  % light q‑jet 1
  \coordinate (J2) at (-12:1.0); % light q‑jet 2
  \coordinate (M)  at (0:0.85);  % merged jet center

  % BACK of large cone
  \def\ang{28}  % full opening angle
  \def\e{0.05}  % eccentricity
  \tikzmath{%
    coordinate \axisvec;
    \axisvec = (O)-(M);
    \x    = veclen(\axisvecx,\axisvecy)*\e*sin(\ang)^2;
    \y    = tan(\ang)*(veclen(\axisvecx,\axisvecy)-\x);
    \a    = veclen(\axisvecx,\axisvecy)*sqrt(\e)*sin(\ang);
    \b    = veclen(\axisvecx,\axisvecy)*tan(\ang)*sqrt(1-\e*sin(\ang)^2);
    \angb = acos(sqrt(\e)*sin(\ang));
  }
  \coordinate (ML) at 
    ($(M)+(-180:\x pt)+(90:\y pt)$);
  % draw back half of the large cone
  \draw[thin,red!40!black,
        top color=red!70!black!60,
        bottom color=red!50!black!70]
    (M) ellipse({\a pt} and {\b pt});

  % JETS: draw three sub‑cones
  \jetcone[green!80!black]{O}{BJ}{14}{0.10} % b‑jet
  %\jetcone{O}{J1}{16}{0.08}                 % q‑jet 1
  %\jetcone{O}{J2}{16}{0.10}                 % q‑jet 2

  % Labels
  \node[green!50!black,align=center,font=\huge]
  at
    ($ (56:1.26) + (13pt,-7pt) $)
  {AK4 Jet\\$(\Delta R = 0.4)$};
  \node[red!80!black,right,align=center, font=\huge] at (0:1.05) {AK8 Jet\\$(\Delta R = 0.8)$};

  % FRONT of large cone
  \draw[thin,red!40!black,fill opacity=0.9,rounded corners=0.001pt,
        top color=red!90!black!40,
        bottom color=red!80!black!50]
    (ML) arc(180-\angb:180+\angb:{\a pt} and {\b pt})
    -- ($(O)-(0.0005,0)$) -- cycle;
\end{tikzpicture}%
      }
      {
        \footnotesize
        \begin{itemize}
          \item CMS jets are clustered with the anti $k_t$ (AK) algorithm.
          \item Jet sizes are defined in terms of the angular separation, $\left(\Delta R = \sqrt{\Delta \phi^{2} + \Delta \eta^{2}}\right)$
        \end{itemize}
      }
    \end{column}
    \begin{column}{0.32\textwidth}
      \centering
      \framesection{Resolved Region}
      \resizebox{0.99\linewidth}{!}{%
      \input{../figures/theory/resolved-region}%
      }
      {
        \footnotesize
        \begin{itemize}
          \item $m_{N} > 0.1 m_{W_{R}}$
          \item Four isolated objects: two leptons and two AK4 jets.
          \item Signal in four-object mass spectrum $m_{\ell \ell j j}$
        \end{itemize}
      }
    \end{column}
    \begin{column}{0.32\textwidth}
      \centering
      \framesection{Boosted Region}
      \resizebox{0.99\linewidth}{!}{%
      \input{../figures/theory/boosted-region}%
      }
      {
        \footnotesize
        \begin{itemize}
          \item $m_{N} < 0.1 m_{W_{R}}$
          \item Two isolated objects: one lepton and one AK8 jet.
          \item Signal in di-object mass spectrum $m_{\ell J}$
        \end{itemize}
      }
    \end{column}
  \end{columns}
\end{frame}

\begin{frame}{Analysis Backgrounds}
  \begin{block}{Dominant Backgrounds}
    The two main backgrounds that result in an $\ell \ell j j$ final state are 
    \boldcol{UMNMaroon}{Drell-Yan (DY)} and \boldcol{UMNMaroon}{$\mathbf{t\bar{t}}$}.
  \end{block}
  \vfill
  \begin{columns}
    \begin{column}{0.5\textwidth}
      \begin{figure}
        \centering
        \input{../figures/theory/dy-process}
        \caption{The Drell-Yan (DY) process.}
      \end{figure}
    \end{column}
    \begin{column}{0.5\textwidth}
      \begin{figure}
        \centering
        \input{../figures/theory/tt-process}
        \caption{$\mathrm{t\bar{t}}$ decay.}
      \end{figure}
    \end{column}
  \end{columns}
  \centering
  \boldcol{UMNSunny}{Minor Backgrounds include W+Jets, Single Top, Diboson, Triboson}
\end{frame}

\subsection{Analysis Regions}

\begin{frame}{Analysis Regions}
  \begin{figure}
    \centering
    \begin{tikzpicture}[scale=5]
  
    % Define the division points for the three regions
    \def\xone{0.45}   % Right boundary of region 1 (SR)
    \def\xtwo{1.05}   % Right boundary of region 2 (CR) / left boundary of region 3 (SB)
    \def\xmax{2}      % Overall width of the box
    
    % Draw and fill the three vertical regions
    \fill [red!20!white]      % Region 1: SR
      (0,0) rectangle (\xone,1);
    \fill [blue!20!white]     % Region 2: CR (Control or Central Region)
      (\xone,0) rectangle (\xtwo,1);
    \fill [green!20!white]    % Region 3: SB
      (\xtwo,0) rectangle (\xmax,1);
      
    % Draw dashed vertical division lines for the three regions
    \draw[dashed,thick] (\xone,0) -- (\xone,1);
    \draw[dashed,thick] (\xtwo,0) -- (\xtwo,1);
    
    % Draw a horizontal dashed line at y = 0.5 (dividing OS and SS regions)
    \draw[dashed,thick] (-0.1,0.5) -- (\xmax,0.5);
    
    % Label the y-axis regions (lepton charges)
    \draw
      (-0.1,0.75) node[anchor=east] {SF}
      (0,0.65) node[anchor=east] {($ee$, $\mu\mu$)}
      (-0.1,0.25) node[anchor=east] {OF}
      (-0.08,0.15) node[anchor=east] {($e\mu$)};
    
    % Label the x-axis regions (lepton isolations)
    \draw
      (\xone,0) node[anchor=north] {150}
      (\xtwo,0) node[anchor=north] {400};
    
    % Add axis descriptions (the Lepton Flavor label remains on the left)
    \draw
      (-0.35,0.5) node[rotate=90,anchor=south] {Lepton Flavor};
    
    % Draw the x-axis as an arrow along the bottom edge.
    \draw[->, thick] (0,0) -- (\xmax+0.2,0) node[anchor=north] {$m_{ll}$ (GeV)};
    \draw[thick] (0,0) -- (0,1) node[anchor=west] {};

    % Add internal region labels (adjust positions as needed)
    \draw
      (\xone/2,0.78) node {DYCR}
      ({\xtwo + (\xmax-\xtwo)/2},0.25) node[align=center] {$e\mu$ Sideband}
      ({\xtwo + (\xmax-\xtwo)/2},0.75) node[align=center] {Signal Region};
    
\end{tikzpicture}
    \caption{A schematic diagram of the analysis region. The DY background is estimated from 
    the DY CR (blue).The backgrounds from $tW$ and $t\bar{t}$ production are estimated from the flavor CR 
    (green), where opposite flavor (OF) leptons are require.}
    \label{fig:dm-mass-scale}
  \end{figure}
\end{frame}

\note[itemize]{
  \item We define these different regions of phase space which we use to help estimate the backgrounds. 
  \item We estimate the backgrounds from Monte Carlo, but we use these different control regions in order to derive 
  corrections to the Monte Carlo in order to better predict the background in our Signal Region.
  \item We can see the Signal Region in the top right. The x-axis is the invariant mass of the two leptons in the final state. 
  So we require that the invariant mass of the two leptons by considerably high in order to reduce the background. 
  \item Moving to the left on the x-axis, you can see that if we lower that dilepton invariant mass requirement, 
  we get closer to the Z peak at 91.2 GeV. So we define a control region here, the Drell-Yan control region, right around 
  the Z peak because this is dominated by Drell-Yan and so we are able to drive corrections to our Drell-Yan Monte Carlo in 
  this control region. 
  \item Going back to our signal region and then moving down the y-axis, we see that in the signal region we require the leptons 
  to have the same flavor. But for ttbar events, the two tops are decaying independently, so they either can decay to the same flavor 
  or opposite flavor. So, when we require that our leptons be opposite flavor, we are in a region that is dominated by ttbar, and has 
  no signal within it either. So we are able to use this region to make more accurate corrections to our ttbar.
}

\section{Background Estimation}

\begin{frame}{Signal Region Backgrounds}
  \begin{columns}
    \begin{column}{0.50\textwidth}
      \centering
      \includegraphics[width=\textwidth]{../figures/plots/sr-bkg-frac-mumu.png}
    \end{column}
    \begin{column}{0.50\textwidth}
        \vspace*{-15mm}
        \centering
        \resizebox{\columnwidth}{!}{%
        \input{../figures/theory/signal-region}%
        }
      \vspace{1ex} 
      \begin{block}{Background Composition}
        As $m_{\ell\ell jj}$ increases, the background becomes more dominated by Drell-Yan (Z+jets).
      \end{block}
    \end{column}
  \end{columns}
\end{frame}

\subsection{$t\bar{t}$ and DY Modeling}

\begin{frame}{Flavor Control Region}
  \begin{columns}
    \begin{column}{0.50\textwidth}
      \centering
      \includegraphics[width=\textwidth]{../figures/plots/mass-fourobject-flavorcr.png}
    \end{column}
    \begin{column}{0.50\textwidth}
        \vspace*{-15mm}
        \centering
        \resizebox{\columnwidth}{!}{%
        \input{../figures/theory/flavor-cr}%
        }
      \vspace{1ex} 
      \begin{block}{$t\bar{t}$ Modeling}
        Simulation of $t\bar{t}$ agrees very well with data in the $e\mu$ sideband.
      \end{block}
    \end{column}
  \end{columns}
\end{frame}

\begin{frame}{Drell-Yan Control Region}
  \begin{block}{Mixed Results}
    DY modeling in the DY CR is mixed - better with leptonic variables and worse with hadronic variables.
  \end{block}
  \vfill
  \begin{columns}
    \begin{column}{0.35\textwidth}
      \includegraphics[width=\textwidth]{%
        ../figures/plots/pt-leadlep-dycr-ee.pdf}
    \end{column}
    \begin{column}{0.35\textwidth}
      \includegraphics[width=\textwidth]{%
        ../figures/plots/pt-leadjet-dycr-ee.png}
    \end{column}
    \begin{column}{0.3\textwidth}
 %     \vspace*{-15mm}
      \centering
      \resizebox{\columnwidth}{!}{%
      \input{../figures/theory/dy-cr}%
      }
    \end{column}
  \end{columns}
\end{frame}

\subsection{Background Reweighting}

\begin{frame}{DY MC Reweighting}
  \begin{columns}
    \begin{column}{0.50\textwidth}
      \centering
      \includegraphics[width=\textwidth]{../figures/plots/lead-jet-pt-reweighting.png}
    \end{column}
    \begin{column}{0.50\textwidth}
        \centering
        \resizebox{0.55\columnwidth}{!}{%
        \input{../figures/theory/dy-cr}
        }
        \vfill
      \begin{block}{}
        \begin{enumerate}
          \item \boldcol{UMNMaroon}{Derive reweighting factors from the dijet mass $m_{jj}$.}
          \begin{itemize}
            \item Scale the DY MC in each bin to agree with data. 
          \end{itemize}
          \item Apply these scale factors to all the DY MC.
          \item Compare Data/MC in the Signal Region.
        \end{enumerate}
        
      \end{block}
    \end{column}
  \end{columns}
\end{frame}

\begin{frame}{DY CR Reweighted}
  \begin{columns}
    \begin{column}{0.69\textwidth}
      \begin{block}{}
        \centering
        The reweighting improves the Z+Jets modeling in the DY control region.
      \end{block}
    \end{column}
    \begin{column}{0.3\textwidth}
      \centering
      \resizebox{0.9\columnwidth}{!}{%
      \input{../figures/theory/dy-cr}
      }
    \end{column}
  \end{columns}
  \begin{columns}
    \begin{column}{0.35\textwidth}
      \begin{figure}
        \includegraphics[width=\textwidth]{%
        ../figures/plots/mlljj-dycr-mumu-pre.png}
        \caption{pre-ratio}
      \end{figure}
    \end{column}
    \begin{column}{0.35\textwidth}
      \begin{figure}
        \includegraphics[width=\textwidth]{%
        ../figures/plots/mlljj-dycr-mumu-post.png}
        \caption{post-ratio}
      \end{figure}
    \end{column}
    \begin{column}{0.3\textwidth}
      \begin{block}{}
        How does reweighting affect the signal region?
          \begin{itemize}
            \footnotesize
            \item Requires unblinding.
          \end{itemize}
      \end{block}
    \end{column}
  \end{columns}
\end{frame}

\section{Looking Forward to Run 3}
\subsection{Previous Results}

\begin{frame}{Signal Region}
  \begin{columns}
    \begin{column}{0.50\textwidth}
      \centering
      \includegraphics[width=\textwidth]{../figures/plots/mlljj-sr-mumu.pdf}
    \end{column}
    \begin{column}{0.50\textwidth}
        \centering
        \resizebox{0.55\columnwidth}{!}{%
        \input{../figures/theory/signal-region.tex}
        }

      \vfill
      \boldcol{UMNMaroon}{The lead jet $p_{T}$ has been reweighted in previous studies.}
      
      \begin{block}{Reweighting with the lead jet $p_T$}
        \begin{itemize}
          \item Before the correction, there is good agreement.
          \item The corrections pulls the MC down, out of agreement.
          \item Excess data isn't very peak-like.
        \end{itemize}
      \end{block}
    \end{column}
  \end{columns}
\end{frame}

\subsection{Optimization Studies}

\begin{frame}{$m_{\ell \ell}$ optimization}
  \begin{columns}
    \begin{column}{0.50\textwidth}
      \centering
      \includegraphics[width=\textwidth]{../figures/plots/mass-fourobject-flavorcr.png}
    \end{column}
    \begin{column}{0.50\textwidth}
 %       \vspace*{-5mm}
        \centering
        \resizebox{\columnwidth}{!}{%
        \input{../figures/theory/signal-region.tex}%
        }
 %     \vspace{1ex} 
      \begin{block}{Defining the signal region}
        Is $m_{\ell \ell} > 400 \mathrm{~GeV}$ the best cut for the signal region?
        \begin{itemize}
          \item Loosening lets in more DY background.
          \item Tightening limits statistical power.
        \end{itemize}
      \end{block}
    \end{column}
  \end{columns}
\end{frame}

\section{Conclusion} % \ssection{Conclusion}

\begin{frame}{Conclusion}
  Many improvements and areas of study under development in the $\mathrm{W_R}$ and $\mathrm{N}$ search:

  \begin{columns} % [t] for top alignment of columns
    %――――――――――――――――――――――――――――
    \begin{column}{0.5\textwidth}
      \begin{block}{Looking Forward to Run 3}
        Beginning comparisons for Run 3 Data/MC.
      \end{block}

      \begin{block}{Futher Optimization Studies}
        \begin{itemize}
          \item Looking at how \boldcol{UMNMaroon}{adding a third jet} to the final‐state objects affects 
          signal and background yields.
        \end{itemize}
      
      \end{block}
    \end{column}

    %――――――――――――――――――――――――――――
    \begin{column}{0.5\textwidth}
      \centering
      \includegraphics[width=0.75\textwidth]{wr-boson-chatgpt.png}
    \end{column}
  \end{columns}
\end{frame}

\begin{backup}

\begin{frame}{LHC}
    \begin{figure}
      \centering
      \includegraphics[width=0.7\textwidth]{../figures/experiment/lhc-schematic.png}
      \caption{Schematic showing the overall layout of the LHC. Image credit -- CERN.}
    \end{figure}
\end{frame}

\end{backup}

\end{document} 
