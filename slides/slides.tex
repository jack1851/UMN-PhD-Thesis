\documentclass[aspectratio=169]{beamer}

\input{../preamble}

%--------------------------------------------------
% Geometry and Scale Settings
% Adjust the scaling factor for physical dimensions.
%--------------------------------------------------
\geometry{mag=1600, truedimen}

%--------------------------------------------------
% University of Minnesota Color Definitions
%--------------------------------------------------
\definecolor{UMNMaroon}{RGB}{122,0,25}
\definecolor{UMNLightGold}{RGB}{255,215,95}
\definecolor{UMNGold}{RGB}{255,204,51}
\definecolor{UMNStormy}{RGB}{64,77,91}
\definecolor{UMNSunny}{RGB}{0,149,182}
\definecolor{UMNLightGray}{RGB}{255,222,122}

%--------------------------------------------------
% Package Imports
% Load additional packages required for the presentation.
%--------------------------------------------------
\usepackage{xspace}         % Ensures \xspace works properly with custom macros.
\usepackage{textpos}        % Provides absolute positioning.
\usepackage{epstopdf}       % Converts EPS images to PDF.
\usepackage{graphicx}       % Handles image inclusion.
\usepackage{beamerthemesplit} % Legacy beamer theme; consider using a modern theme if desired.
\usepackage{soul}           % Allows text highlighting and underlining.
\usepackage{xifthen}        % Enables conditional statements.
\usepackage{listings}       % For code formatting.
\usepackage{hyperref}       % For hyperlink support in the generated PDF.
\usepackage{bbding}         % Provides additional symbols (e.g., checkmarks).

%--------------------------------------------------
% Hyperref Configuration
% Customize link colors and PDF metadata.
%--------------------------------------------------
\hypersetup{
    colorlinks=true,
    linkcolor=UMNMaroon,
    urlcolor=UMNStormy,
    citecolor=UMNSunny,
    pdftitle={Presentation Title},
    pdfauthor={Your Name}
}

%--------------------------------------------------
% Listings Settings for Code Snippets
%--------------------------------------------------
\definecolor{backcolour}{rgb}{0.95,0.95,0.92}
\lstdefinestyle{mystyle}{
    backgroundcolor=\color{UMNLightGray},
    commentstyle=\color{UMNStormy},
    keywordstyle=\color{UMNMaroon},
    numberstyle=\tiny\color{UMNMaroon},
    stringstyle=\color{UMNSunny},
    basicstyle=\ttfamily\footnotesize,
    emph={ldmx },
    emphstyle=\color{UMNMaroon},
    breakatwhitespace=false,
    breaklines=true,
    captionpos=b,
    keepspaces=true,
    numbers=left,
    numbersep=5pt,
    showspaces=false,
    showstringspaces=false,
    showtabs=false,
    tabsize=2
}
\lstset{style=mystyle}

%--------------------------------------------------
% Custom Commands and Environments
%--------------------------------------------------

%--- Code Listing Commands ---%
% Import code from an external file.
% #1: Options for the formatter.
% #2: File path to the code file.
\newcommand{\codefile}[2]{%
    \lstinputlisting[#1]{#2}
}%

% Inline code command with a styled box.
% #1: Inline code text.
\newcommand{\code}[1]{%
  \tikz[baseline=(codebox.base)]{
    \node[inner sep=1pt,outer sep=0pt,draw=UMNStormy,line width=0.05mm,fill=UMNLightGray,rounded corners=0.03cm] (codebox) 
      {\textcolor{UMNSunny}{#1}};
  }%
}%

% Styled title command within a frame for subsection headings.
% #1: Text to style.
\newcommand{\framesection}[1]{%
  \boldcol{UMNStormy}{#1}\hspace{1em}%
}%

%--- List Item Symbols ---%
% Define checkmark and crossmark symbols with custom colors.
\def\checkmark{{\color{UMNSunny}\Checkmark}}
\def\crossmark{{\color{UMNMaroon}\XSolidBrush}}

% Define "done" and "todo" list items.
\def\done{\item[$\color{UMNMaroon}\boxtimes$]}
\def\todo{\item[$\color{UMNMaroon}\square$]}

% Define "good" and "bad" list items.
\def\good{\item[\checkmark]}
\def\bad{\item[\crossmark]}

%--- Hyperlink Button ---%
% Creates a clickable button linking to a URL.
% #1: URL to link to.
% #2: Button text.
\newcommand{\hlink}[2]{%
    \href{#1}{\beamergotobutton{#2}}
}%

%--- Bold and Colored Text ---%
% Boldens and colors text.
% #1: Color.
% #2: Text.
\newcommand{\boldcol}[2]{%
    {\color{#1}\textbf{#2}}\xspace
}%

%--- One Plot Slide ---%
% Creates a slide with one plot that fills most of the frame.
% #1: Slide Title.
% #2: Slide Subtitle.
% #3: File path to the image.
\newcommand{\oneplotslide}[3]{%
    \begin{frame}
        \frametitle{#1}
        \framesubtitle{#2}
        \begin{figure}
            \centering
            \includegraphics[width=\textwidth,height=0.8\textheight,keepaspectratio]{#3}
        \end{figure}
     \end{frame}
}%

%--- Four Tiles Slide ---%
% Arranges content into four tiles on a slide.
% #1: Slide Title.
% #2: Slide Subtitle.
% #3: Top left content.
% #4: Top right content.
% #5: Bottom left content.
% #6: Bottom right content.
\newcommand{\fourtileslide}[6]{%
    \begin{frame}
        \frametitle{#1}
        \framesubtitle{#2}
        \begin{figure}[h]
            \centering
            \begin{tabular}{cc}
                #3 & #4 \\
                #5 & #6
            \end{tabular}
        \end{figure}
    \end{frame}
}%

%--- Four Plots Slide ---%
% A slide that tiles four plots; empty paths yield a blank tile.
% #1: Slide Title.
% #2: Slide Subtitle.
% #3: Top left image file path.
% #4: Top right image file path.
% #5: Bottom left image file path.
% #6: Bottom right image file path.
\newcommand{\fourplotslide}[6]{%
    \fourtileslide{#1}{#2}%
        {\ifthenelse{\isempty{#3}}{}{\includegraphics[height=0.4\textheight]{{#3}}}}%
        {\ifthenelse{\isempty{#4}}{}{\includegraphics[height=0.4\textheight]{{#4}}}}%
        {\ifthenelse{\isempty{#5}}{}{\includegraphics[height=0.4\textheight]{{#5}}}}%
        {\ifthenelse{\isempty{#6}}{}{\includegraphics[height=0.4\textheight]{{#6}}}}%
}%

%--- Figure with Comments ---%
% Creates a slide with an image on the left and commentary on the right.
% #1: File path of the image.
\newenvironment{figurecomments}[1]{%
  \begin{columns}
    \begin{column}{0.6\textwidth}
      \begin{figure}
        \centering
        \includegraphics[height=0.7\textheight]{#1}
      \end{figure}
    \end{column}
    \begin{column}{0.4\textwidth}
      \footnotesize
}{%
    \end{column}
  \end{columns}
}%

%--- Section Title Slide ---%
% Creates a full-slide title card for sections.
% #1: Section title.
\newcommand{\sectionframe}[1]{%
\begin{frame}
    \vfill
    \centering
    \begin{beamercolorbox}[sep=8pt,center,shadow=true,rounded=true]{title}
        \usebeamerfont{title}#1\par%
    \end{beamercolorbox}
    \vfill
\end{frame}
}%

%--- Backup Slides Environment ---%
% Wrap backup slides to exclude them from the main slide count.
% Optionally, you can title the backup section (default is "Questions").
\newenvironment{backup}[1][Questions]{%
  \sectionframe{#1}
  \newcounter{finalframe}
  \setcounter{finalframe}{\value{framenumber}}
}{%
  \setcounter{framenumber}{\value{finalframe}}
}%

\usepackage{pgfpages}
\setbeamertemplate{note page}[plain]
\setbeameroption{show notes}
\addtobeamertemplate{note page}{}{\thispdfpagelabel{notes:\insertframenumber}}

\newcommand{\ssection}[1]{%
  \section{#1}%
  \sectionframe{#1}%
}%

\newenvironment{sframe}[1]{%
  \subsection{#1}%
  \begin{frame}{#1}%
}{%
  \end{frame}%
}%

\title[$\mathrm{W_R}$ Boson and Heavy Neutrino Search]{%
  Search for a Right-Handed W Boson and Heavy Neutrino of the Left-Right Symmetric Standard Model%
}

\begin{document}

\begin{frame}
  \maketitle
\end{frame}

\note[itemize]{
\item Hello everyone, for those of you who don't know me, my name is Billy.
  I am one of Jeremy's graduate students, and in this seminar I am going to
  talk about our search for a right-handed W boson and heavy neutrino of the 
  Left-Right Symmetric Standard Model. 
}

\begin{frame}{Outline}
  \begin{columns}
    \begin{column}{0.5\textwidth}
      \tableofcontents
    \end{column}
    \begin{column}{0.5\textwidth}
      \includegraphics[width=0.9\textwidth]{../figures/experiment/cms-slice}
      \\
      \resizebox{0.9\textwidth}{!}{\begin{tikzpicture}
    \begin{feynman}
      %
      % -- Define the incoming quarks on the left
      \vertex (qbar) at (-2, 1.5) {\(\bar{q}'\)};
      \vertex (q)    at (-2,-1.5) {\(q\)};
      
      % -- Central vertex where q qbar' -> W_R
      \vertex [dot] (wr) at (0,0) {};
      
      % -- First decay: W_R -> N_\ell + l
      \vertex [dot] (Nl) at (2.5,0) {};
      \vertex (l1)  at (4,-1.5) {\(\ell^{+}\)};
      \vertex (N) at (4, 1.5) {};
      
      % -- Second decay: N_\ell -> l + W_R
      %    Then W_R -> 2 jets
      \vertex [dot] (wrl) at (4, 1.5) {};
      \vertex (l2) at (5.5,  3) {\(\ell^{-}\)};
      \vertex (wr2) at (6,  1) {};

      %    Then W_R -> 2 jets
      \vertex [dot] (jj) at (6, 1) {};
      \vertex (jet1) at (7.5,  1.7) {jet};
      \vertex (jet2) at (7.5, 0.3) {jet};
      
      % -- Draw the diagram
      \diagram*{
        % Incoming quarks
        (qbar) -- [fermion] (wr) -- [fermion] (q),
        
        % First decay of W_R
        (wr) -- [boson, edge label=\(W_R\)] (Nl),
        (l1) -- [fermion] (Nl),
        (Nl) -- [fermion, edge label=\(N_\ell\)] (wrl),
        
        % Decay of heavy neutrino N_l
        (wrl) -- [fermion] (l2),
        (wrl) -- [boson, edge label=\(W_R^{*}\)] (wr2),

        % Decay of second W_R into jets
        (jet1) -- [fermion] (jj) -- [fermion] (jet2),
      };
    \end{feynman}
  \end{tikzpicture}}
    \end{column}
  \end{columns}
\end{frame}

\note[itemize]{
\item I will begin with some background, and explain what exactly we mean when we 
  say the weak force is left-handed and we are searching for a right-handed W and 
  right-handed neutrino. 
\item I will then discuss the class of models, called left-right symmetric models, 
    that adds these particles to the standard model.
\item We search for these particles using data from the CMS detector at the Large 
  Hadron Collider, most people here familiar with these to some degree so I'll quickly
  describe them.
\item I will explain the general search strategy that we use to look for these particles 
  using CMS data. 
\item And then I will discuss the backgrounds and how we are estimating them.
\item Lastly, I'll end with a discussion about some of the things that we are looking 
  forward to in Run 3.
}

\section{Background}

\subsection{Weak Interactions and Handedness}

\begin{frame}{Parity Violation in $\beta$ Decay}
  \begin{itemize}
    \item The parity operator $\hat{P}$ corresponds to a discrete 
      transformation $x \rightarrow -x$, etc.
    \item In 1957 C.S. Wu studied the beta decay:
      $\ce{^{60}_{27}Co \rightarrow ^{60}_{28}Ni + e^{-} + \bar{\nu_{e}}}$
  \end{itemize}
  \begin{columns}
    \begin{column}{0.65\textwidth}
      \begin{figure}
        \centering
        \includegraphics[width=\textwidth]{wu-experiment.png}
      \end{figure}
    \end{column}
    \begin{column}{0.3\textwidth}
      \begin{block}{}
        If parity were conserved: expect equal rate for producing $e^{-}$ in directions
            along and opposite to the nuclear spin.
      \end{block}
    \end{column}
  \end{columns}
  \begin{itemize}
    \item Observed \boldcol{UMNMaroon}{electrons emitted preferentially} in direction 
      opposite to applied field.
  \end{itemize}
  \begin{block}{}
    \centering
      Conclude \boldcol{UMNSunny}{parity is violated} in weak interactions.
  \end{block}
\end{frame}

\note[itemize]{
\footnotesize
\item In the year 1957, a physicist from Colombia University named Wu conducted an experiment 
  to test the mirror symmetry of weak interactions. Mirror symmetry, or parity invariance, is 
  the idea that the mirror image of any physical process also represents a perfectly possible 
  physical process.
\item Prior to 1957, parity invariance had been confirmed for strong and electromagnetic processes, 
  and it was widely assumed that the same would apply for the weak force, but there was no 
  experimental confirmation of this. 
\item So, Wu conducted this experiment where she aligned cobalt 60 nuclei in a strong magnetic field. 
  The polarized cobalt nuclei then beta in a decayed in a weak interaction to nickel, an electron 
  and antineutrino, and Wu recorded the directions of the emitted electrons. 
\item Because both B and $\mu$ are axial vectors, they do not change sign under the parity transformation. 
  Hence when viewed in the parity inverted “mirror”, the only quantity that changes sign is the vector 
  momentum of the emitted electron. Hence, if parity were conserved in the weak interaction, the rate 
  at which electrons were emitted at a certain direction relative to the B-field would be identical to 
  the rate in the opposite direction. And this is what they expected to find.
\item Experimentally, it was observed that more electrons were emitted in the hemisphere opposite to the 
  direction of the applied magnetic field than in the hemisphere in the direction of the applied field, 
  thus providing a clear demonstration that parity is NOT conserved in the weak interaction. The result 
  was surprising to physicists, with Wolfgang Pauli lamenting, “I cannot believe that God is a weak 
  left-hander”.
}

\begin{frame}{Helicity States}
  \begin{block}{Definition}
    Define the \boldcol{UMNSunny}{Helicity} of a particle as the projection of its spin onto the direction of momentum.
  \end{block}
  \begin{columns}
    \begin{column}{0.60\textwidth}
      \begin{figure}
        \centering
        \input{../figures/theory/helicity.tex}
        \caption{Illustration of helicity states in muon decay. 
          $e^{+}$ and $\bar{\nu}_{\mu}$ are emitted as \boldcol{UMNMaroon}{right-handed} helicity states. 
          $\nu_e$ is emitted as a \boldcol{UMNMaroon}{left-handed} helicity state.}
      \end{figure}
    \end{column}
    \begin{column}{0.40\textwidth}
      \begin{block}{A Good Quantum Number}
        \centering
        $[\hat{H}_D, \hat{S} \cdot \hat{p}] = 0$
        \begin{itemize}
          \item Possible to find solutions of the Dirac equation which are also 
            eigenstates of Helicity.
        \end{itemize}
      \end{block}
    \end{column}
  \end{columns}
  \centering
  \boldcol{UMNSunny}{But it is important to remember that helicity is not Lorentz invariant.}
\end{frame}

\note[itemize]{
\footnotesize
\item In order to understand exactly what I mean by left-handed and right-handed bosons, it is important 
  to understand how spinors interact with the weak force. Spinors are solutions of the Dirac equation, 
  which describes relativistic spin $\frac{1}{2}$ particles. 
\item Its convenient to define spinors in terms of their helicity, which is the projection of the particles 
  spin along its direction of momentum, which we see in the image on the right here.
\item We do this because the helicity operator commutes with the dirac Hamiltonian, so it is possible to 
  identify spinor states which are simultaneous eigenstates of the free particle dirac Hamiltonian and the 
  helicity operator. And in fact, the motivation to introduce helicity is that the spin operator Sz does not 
  commute with the Hamiltonian, so we can’t just say that one solution of the dirac equation is a spin up 
  particle and another is a spin down particle, for instance. But, the dot product of the spin and momentum 
  operators is a quantity that does commute with the Hamiltonian.
\item For a spin half particle, the component of spin measured along any axis is quantized to be +/- 1/2, and 
  so the eigenvalues of the helicity operator acting on a spinor are also plus or minus 1/2. The two possible 
  helicity states for a spin- half fermion are termed right-handed, which is when the spin and momentum are 
  aligned, as shown here, and left-handed, which is when the  spin and momentum are antialigned, as shown in 
  the image on the right.
\item Then we have four solutions of the dirac equation, two right and left-handed helicity particle spinors, 
  and two right and two right and left-handed antiparticle spinors. 
\item Unfortunatley, however, heliciy has a kryptonite, and it is that it is not Lorentz invariant. This is 
  because for particles with mass, it is always possible to boost into a frame in which the direction of the 
  particle's momentum is reversed and therefore the helicity flips. And because of this, helicity cannot be a 
  fundamental property of particles.
}

\begin{frame}{Chiral States}
  \begin{itemize}
    \item \boldcol{UMNSunny}{Chirality}: The Lorentz invariant ``version'' of helicity and an intrinsic property of particles.
      \begin{itemize}
        \item In general, define the eigenstates of $\gamma^{5}$ as \boldcol{UMNMaroon}{left} and \boldcol{UMNMaroon}{right-handed chiral} states
      \end{itemize}
    \item \boldcol{UMNSunny}{Ultra-relativistic limit}:
      $E >> m$
      \begin{itemize}
        \item Chiral eigenstates correspond to helicity eigenstates.
      \end{itemize}
    \item \boldcol{UMNSunny}{Projection Operators}
      \begin{itemize}
        \item Project out the chiral eigenstates:
      \end{itemize}
      $$
        \boxed{P_{L} = \frac{1}{2}\left(1 - \gamma^{5}\right) \quad P_{R} = \frac{1}{2}\left(1 + \gamma^{5}\right)}
      $$
    \item \boldcol{UMNSunny}{Spinor Decomposition}
      \begin{itemize}
        \item Can write any spinor in terms of its left and right-handed chiral components
      \end{itemize}
      $$
        \Psi = \Psi_{L} + \Psi_{R} = P_{L}\Psi + P_{R}\Psi = \frac{1}{2}\left(1 - \gamma^{5}\right)\Psi + \frac{1}{2}\left(1 + \gamma^{5}\right)\Psi
      $$
  \end{itemize}
\end{frame}

\note[itemize]{
\footnotesize
\item Fortunately, there is a Lorentz invariant version of helicity, which is called chirality. I say 
  "version" in quotes because while chirality is an intrinsic property of a particle, unlike helicity does 
  not have a simple physical interpretation. 
\item In general, the eigenstates of the gamma 5-matrix are defined as left- and right-handed chiral states 
  where I have denoted with subscripts L and R to distinguish them from the left and right handed helicity 
  eigenstates on the previous slide.
\item The link between helicity and chirality appears in the massless limit. In this limit, and only in this 
  limit, helicity eigenstates are also eigenstates of the gamma 5 matrix. In other words, in the massless limit, 
  helicity and chirality eigenstates are the same. 
\item Now any Dirac spinor can be decomposed into left- and right-handed chiral components using these two chiral 
  projection operators PL and PR. PR projects out right-handed chiral particle states and left handed chiral 
  antiparticle states. And PL projects out left-handed chiral particle states and right-handed chiral antiparticle 
  states. And since PR and PL project out chiral states, any general spinor u can be decomposed into left and 
  right-handed chiral components, and you can see this naively if you add PL and PR together you get 1.
}

\subsection{Left-Right Symmetric Extensions}

\begin{frame}{Charged Weak Interactions}
  \begin{columns}
    \begin{column}{0.5\textwidth}
      \begin{figure}
        \centering
        \input{../figures/theory/left-w-vertex}
        \caption{The SM left-handed charged weak vertex.}
      \end{figure}
      \begin{itemize}
        \item Only \boldcol{UMNMaroon}{left-handed chiral} components of 
          particle spinors participate in charged weak interactions.
        \item No theoretical explanation as to \emph{why} 
          the weak force is left-handed.
      \end{itemize}
    \end{column}
    \begin{column}{0.5\textwidth}
      \begin{figure}
        \centering
        \input{../figures/theory/right-w-vertex}
        \caption{The LRSM right-handed weak vertex.}
      \end{figure}
      \begin{itemize}
        \item Introduce $W_{R}^{\pm}$ bosons to couple to \boldcol{UMNMaroon}{right-handed chiral} components of 
        particle spinors.
        \item Under parity, left and right-handed vertices transform into each other.
      \end{itemize}
    \end{column}
  \end{columns}
\end{frame}

\note[itemize]{
\footnotesize
\item If this left-handed chiral projection operator looks familiar, that is because is included in the 
  charged weak interaction vertex. What this means is that the weak force projects out the left-handed 
  chiral state, so that only left-handed chiral particle states participate in the charged-current 
  weak interaction. 
\item This is what we really mean when we say the weak force is 'left-handed.' The different coupling 
  of the weak charged-current interaction to left- handed and right-handed chiral states is the origin 
  of parity violation. 
\item But there is no theoretical reason behind this. There's nothing in the Standard Model that tells 
  us that the charged weak interaction vertex should include the left chiral projection operator, instead 
  of the right chirarl projection operator, or no projection operator at all, which is the case for QED 
  and QCD. This is just something that has been determined from experiment. 
\item The weak interaction symmetry group is called SU(2) left.
\item In this group left-chiral fermions form doublets of weak isospin, where each entry is really a dirac 
  wavefunction, so for example eL is really the left chiral component of the electrons wavefunction. Weak 
  isospin symmetry is very much like the isospin symmetry for the strong interactions. For example if we 
  look at the leptons, the neutrino and lepton have a total weak isospin of 1/2 and a third component of 
  weak isospin of +1/2 for the neutrino and minus 1/2 for the electron. So a left-handed electron wavefunction, 
  for example, emits a W boson and continues on its way. In the process it flips from an electron state to 
  a neutrino state, analogous to way a spin $\frac{1}{2}$ particle flips its spin when it emits a photon.
\item Experiment also indicates that all neutrinos are left-handed. So the right-handed leptons are singlets 
  in this group. Physically, this means that they don’t couple to the charged weak bosons.
\item The left-chiral quarks are similarly placed in doublets and the right chiral quarks in singlets. 
}

\begin{frame}{Left-Right Symmetric Models}
  \begin{block}{Postulate of Left-Right Symmetric Models}
    The weak force's left-handed nature is a low energy phenomenon -- 
    the result of a broken left-right symmetry that is restored at a multi-TeV energy scale.
  \end{block}
  \vfill
  \begin{table}
    \begin{tabular}{lcc}
    \toprule
    & \textbf{Electroweak Standard Model} 
    & \textbf{Left-Right Symmetric Model} \\ 
    \midrule
    \textbf{Gauge Group} 
      & \(\mathrm{SU}(2)_{L} \times \mathrm{U}(1)\) 
      & \(\textcolor{red}{\mathrm{SU}(2)_{R}} \times \mathrm{SU}(2)_{L} \times \mathrm{U}(1)\) \\[6pt]
    \textbf{Fermions} 
      & \(Q_{L} = (u_i,\,d_i)_{L},\quad L_{L} = (\ell_i,\,\nu_i)_{L}\) 
      & \(Q_{R} = (u_i,\,d_i)_{R},\quad L_{R} = (\ell_i,\,n_i)_{R}\) \\[6pt]
    \textbf{Gauge Bosons} 
      & \(W_{L}^{\pm},\ Z,\ \gamma\) 
      & \(W_{R}^{\pm},\ Z,\ Z'\) \\ 
    \bottomrule
  \end{tabular}
    \caption{Summary of the Left-Right Symmetric SM
    with new extensions in \textcolor{red}{red}.}
  \end{table}
\end{frame}

\note[itemize]{
\footnotesize
\item Left-right symmetric extensions restore left-right symmetry to the standard model by including 
  a SU(2)R group in addition to the usual SU(2)L. 
\item Just as the left-chiral fermions are doublets under SU(2)L we will take the right-chiral 
  fermions to be doublets under SU(2)R. In particular notice that we must add right-chiral neutrinos 
  to go with the right- leptons so that they can form doublets of SU(2)R.
\item These right-chiral fermions then have charged current gauge interactions mediated by the charged 
  gauge bosons of SU(2)R, which we can call WR±. 
\item Then, the interactions of left-chiral and right-chiral fermions are identical and therefore parity 
  is conserved. This is why this model is called the left-right symmetric model. 
\item Lastly, charged current interactions mediated by the right handed W have a coupling constant given 
  by gR. This is a free parameter and so could take on any value, but we assume that it is equal to the 
  left-handed weak coupling constant since this would be the most symmetric model.
}

\begin{frame}{Seesaw Mechanism}
  \begin{figure}
    \centering
    \includegraphics[width=0.70\textwidth]{seesaw-mechanism.png}
    \caption{In LRSM models the lightness of SM neutrinos is explained
    via the seesaw mechanism. Artwork by Sandbox Studio, Chicago with Ana Kova.}
  \end{figure}
\end{frame}

\begin{frame}{Feynman Diagram}
  \begin{figure}
    \centering
    \begin{tikzpicture}
    \begin{feynman}
      %
      % -- Define the incoming quarks on the left
      \vertex (qbar) at (-2, 1.5) {\(\bar{q}'\)};
      \vertex (q)    at (-2,-1.5) {\(q\)};
      
      % -- Central vertex where q qbar' -> W_R
      \vertex [dot] (wr) at (0,0) {};
      
      % -- First decay: W_R -> N_\ell + l
      \vertex [dot] (Nl) at (2.5,0) {};
      \vertex (l1)  at (4,-1.5) {\(\ell^{+}\)};
      \vertex (N) at (4, 1.5) {};
      
      % -- Second decay: N_\ell -> l + W_R
      %    Then W_R -> 2 jets
      \vertex [dot] (wrl) at (4, 1.5) {};
      \vertex (l2) at (5.5,  3) {\(\ell^{-}\)};
      \vertex (wr2) at (6,  1) {};

      %    Then W_R -> 2 jets
      \vertex [dot] (jj) at (6, 1) {};
      \vertex (jet1) at (7.5,  1.7) {jet};
      \vertex (jet2) at (7.5, 0.3) {jet};
      
      % -- Draw the diagram
      \diagram*{
        % Incoming quarks
        (qbar) -- [fermion] (wr) -- [fermion] (q),
        
        % First decay of W_R
        (wr) -- [boson, edge label=\(W_R\)] (Nl),
        (l1) -- [fermion] (Nl),
        (Nl) -- [fermion, edge label=\(N_\ell\)] (wrl),
        
        % Decay of heavy neutrino N_l
        (wrl) -- [fermion] (l2),
        (wrl) -- [boson, edge label=\(W_R^{*}\)] (wr2),

        % Decay of second W_R into jets
        (jet1) -- [fermion] (jj) -- [fermion] (jet2),
      };
    \end{feynman}
  \end{tikzpicture}
    \caption{$\mathrm{W_R}$ decay chain. 
    The final state observables are two same-flavor leptons
    and two jets.}
  \end{figure}
\end{frame}

\note[itemize]{
  \item This feynmann diagram gives us a sense of what exactly we are looking for. 
  \item We have two qaurks coming in, annihilating and producing a right-handed W. 
  That right-handed W decays to a lepton and a heavy neutrino. That heavy neutrino 
  then decays back into a same flavor lepton and an off-shell right handed W, which 
  then decays into two quarks which hadronize to jets. 
  \item So, the observable objects in the decay are two same flavor leptons, and two jets. 
}

\section{LHC and CMS}

\begin{frame}{LHC}
  \begin{figure}
    \centering
    \includegraphics[width=0.7\textwidth]{../figures/experiment/lhc-schematic.png}
    \caption{Schematic showing the overall layout of the LHC. Image credit -- CERN.}
  \end{figure}
\end{frame}

\note[itemize]{
  \item The Large Hadron Collider is the worlds most powerful particle accelerator. 
  \item The accelerator sits 100 meters underground at CERN in Geneva, and consists 
    of a 27 kilometer ring of superconducting magnets. 
  \item It accelerates beams of hadrons in opposite directions to nearly the speed 
    of light, which are made to collide at four interaction points with a center of 
    mass energy of 13 TeV.
  \item One of these interaction points is at the Compact Muon Solenoid, or CMS 
    detector, which is what we are using to look for the WR.
}

\begin{frame}{CMS Detector}
  \begin{figure}
    \centering
    \includegraphics[width=0.85\textwidth]{../figures/experiment/cms-slice.png}
    \caption{CMS detector (slice view). Image credit -- CERN.}
  \end{figure}
\end{frame}

\note[itemize]{
  \item The CMS detector is a general-purpose detector designed to observe any 
    new physics that the LHC might reveal.
  \item The detector is shaped like a cylindrical onion, with several concentric 
    layers of components. 
  \item Starting from the center, the silicon tracker identifies the paths taken 
    by charged particles. 
  \item The electromagnetic calorimeter, or ECAL, measures the energy of electrons 
    and photons by stopping them.
  \item Hadrons fly through the ECAL and are stopped by the next layer, the 
    hadronic calorimeter or HCAL.
  \item Beyond the HCAL is the a superconducting solenoid, which is the worlds most 
    powerful solenoid magnet ever made and generates a 3.8 tesla magnetic field. 
  \item At the outermost layer we have iron return yoke from the solenoid interspersed 
    with muon chambers. Muons are not stopped by the calorimeters so these special sub 
    detectors have to be built to detect them. 
}

\section{Search Strategy} % \ssection{Search Strategy}

\subsection{Resonance Searches}

\begin{frame}{Resonance Search}
  \begin{columns}
    \begin{column}{0.4\textwidth}
      \vfill
      {The cross section a \boldcol{UMNSunny}{Breit-Wigner distribution}:
      $$
        \sigma \propto \frac{1}{\left(m_{lljj}^{2} - m_{W_R}^{2}\right)^{2} + \Gamma^{2} m^{2}_{W_R}}
      $$}
      \begin{figure}
        \centering
        \input{../figures/theory/resonance}
      \end{figure}
    \end{column}
    \begin{column}{0.59\textwidth}
      \begin{block}{Bump Huntin'}
        Search for an excess of events in the \boldcol{UMNMaroon}{four object invariant mass} 
        spectrum consisting of the two leptons and two jets ($m_{\ell \ell j j}$).
      \end{block}
      \begin{figure}
        \centering
        \resizebox{0.9\linewidth}{!}{%
          \input{../figures/theory/wr-decay-annotated}%
      }
      \end{figure}
    \end{column}
  \end{columns}
\end{frame}

\note[itemize]{
  \item The general strategy that we use to search for the right-handed W and heavy 
    neutrino is the classic resonance search, otherwise known as bump hunting. 
  \item The way that this works is that we add up the four momentum of our final state 
    particles and square that to get the four object invariant mass of our final state particle.
  \item Then by conservation of energy and momentum, the four momentum of the right-handed W is 
    equal to the total four momentum of our observables, which gives us the relation q squared 
    equals mlljj squared. 
  \item The W propagator then has a factor of 1 over q squared minus mW squared in the matrix element,
   where mW is the on shell mass of the right handed W and this is proportional to the interaction cross 
   section.  So, when the four object mass of our final state particles approaches the mass of the 
   right-handed W, we observe a resonance in the invariant mass spectrum. 
  \item Perhaps the most famous example of this is the decay of the Higgs to two photons. Here we have a 
    plot of the invariant mass spectrum of the diphotons, and when that mass approaches the mass of the 
    Higgs, we observe a bump in the spectrum.
  \item And I should say that the reason we favor this particular decay mode, is that we can also use this 
    same technique to measure the mass of the heavy neutrino by looking at the ljj mass spectrum. So in a 
    sense we can kill two birds with one stone by examining this particular decay mode. 
}

\begin{frame}[t]{Analysis Topologies}
  \begin{columns}[t]
    \begin{column}{0.32\textwidth}
      \centering
      \framesection{CMS Jets}
      \resizebox{0.99\linewidth}{!}{%
      % Author: Izaak Neutelings (May 2021)
% Description: hadronic top quark jet
\usetikzlibrary{calc,math,decorations.pathreplacing}

% TikZ arrow head
\tikzset{>=latex}

% COLORS
\colorlet{myblue}{blue!70!black}
\colorlet{mydarkblue}{blue!40!black}
\colorlet{mygreen}{green!40!black}
\colorlet{myred}{red!65!black}

% Cone styles
\tikzstyle{cone}     =[thin,blue!50!black,fill=blue!50!black!30]
\tikzstyle{conebase} =[cone,fill=blue!50!black!50]

% Macro to draw a jet cone from point #2 back to #1
% #1 (optional): color
% #2: apex coordinate name
% #3: base coordinate name
% #4: full opening angle (in degrees)
% #5: eccentricity factor
\newcommand\jetcone[5][blue]{{
  \pgfmathanglebetweenpoints%
    {\pgfpointanchor{#2}{center}}%
    {\pgfpointanchor{#3}{center}}%
  \edef\ang{#4/2}       % half‑opening angle
  \edef\e{#5}           % eccentricity
  \edef\vang{\pgfmathresult} % direction of axis
  \tikzmath{%
    coordinate \conevec;
    \conevec = (#2)-(#3);
    \x    = veclen(\conevecx,\conevecy)*\e*sin(\ang)^2;
    \y    = tan(\ang)*(veclen(\conevecx,\conevecy)-\x);
    \a    = veclen(\conevecx,\conevecy)*sqrt(\e)*sin(\ang);
    \b    = veclen(\conevecx,\conevecy)*tan(\ang)*sqrt(1-\e*sin(\ang)^2);
    \angb = acos(sqrt(\e)*sin(\ang));
  }
  \coordinate (tmpL) at 
    ($(#3)-(\vang:\x pt)+(\vang+90:\y pt)$);
  % Draw full ellipse for cone base
  \draw[thin,#1!40!black,rotate=\vang,
        top color=#1!50!black!80,
        bottom color=#1!40!black!80,
        shading angle=\vang]
    (#3) ellipse({\a pt} and {\b pt});
  % Draw the “cone side” as an arc + lines back to apex
  \draw[thin,#1!40!black,rotate=\vang,rounded corners=0.001pt,
        top color=#1!90!black!20,
        bottom color=#1!50!black!50,
        shading angle=\vang]
    (tmpL) arc(180-\angb:180+\angb:{\a pt} and {\b pt})
    -- (#2) -- cycle;
}}

\begin{tikzpicture}[scale=7]
  % Core points
  \coordinate (O)  at (0,0);
  \coordinate (BJ) at (56:1.1);  % b‑jet axis
  \coordinate (J1) at (12:1.0);  % light q‑jet 1
  \coordinate (J2) at (-12:1.0); % light q‑jet 2
  \coordinate (M)  at (0:0.85);  % merged jet center

  % BACK of large cone
  \def\ang{28}  % full opening angle
  \def\e{0.05}  % eccentricity
  \tikzmath{%
    coordinate \axisvec;
    \axisvec = (O)-(M);
    \x    = veclen(\axisvecx,\axisvecy)*\e*sin(\ang)^2;
    \y    = tan(\ang)*(veclen(\axisvecx,\axisvecy)-\x);
    \a    = veclen(\axisvecx,\axisvecy)*sqrt(\e)*sin(\ang);
    \b    = veclen(\axisvecx,\axisvecy)*tan(\ang)*sqrt(1-\e*sin(\ang)^2);
    \angb = acos(sqrt(\e)*sin(\ang));
  }
  \coordinate (ML) at 
    ($(M)+(-180:\x pt)+(90:\y pt)$);
  % draw back half of the large cone
  \draw[thin,red!40!black,
        top color=red!70!black!60,
        bottom color=red!50!black!70]
    (M) ellipse({\a pt} and {\b pt});

  % JETS: draw three sub‑cones
  \jetcone[green!80!black]{O}{BJ}{14}{0.10} % b‑jet
  %\jetcone{O}{J1}{16}{0.08}                 % q‑jet 1
  %\jetcone{O}{J2}{16}{0.10}                 % q‑jet 2

  % Labels
  \node[green!50!black,align=center,font=\huge]
  at
    ($ (56:1.26) + (13pt,-7pt) $)
  {AK4 Jet\\$(\Delta R = 0.4)$};
  \node[red!80!black,right,align=center, font=\huge] at (0:1.05) {AK8 Jet\\$(\Delta R = 0.8)$};

  % FRONT of large cone
  \draw[thin,red!40!black,fill opacity=0.9,rounded corners=0.001pt,
        top color=red!90!black!40,
        bottom color=red!80!black!50]
    (ML) arc(180-\angb:180+\angb:{\a pt} and {\b pt})
    -- ($(O)-(0.0005,0)$) -- cycle;
\end{tikzpicture}%
      }
      {
        \footnotesize
        \begin{itemize}
          \item CMS jets are clustered with the anti $k_t$ (AK) algorithm.
          \item Jet sizes are defined in terms of the angular separation, $\left(\Delta R = \sqrt{\Delta \phi^{2} + \Delta \eta^{2}}\right)$
        \end{itemize}
      }
    \end{column}
    \begin{column}{0.32\textwidth}
      \centering
      \framesection{Resolved Region}
      \resizebox{0.99\linewidth}{!}{%
      \input{../figures/theory/resolved-region}%
      }
      {
        \footnotesize
        \begin{itemize}
          \item $m_{N} > 0.1 m_{W_{R}}$
          \item Four isolated objects: two leptons and two AK4 jets.
          \item Signal in four-object mass spectrum $m_{\ell \ell j j}$
        \end{itemize}
      }
    \end{column}
    \begin{column}{0.32\textwidth}
      \centering
      \framesection{Boosted Region}
      \resizebox{0.99\linewidth}{!}{%
      \input{../figures/theory/boosted-region}%
      }
      {
        \footnotesize
        \begin{itemize}
          \item $m_{N} < 0.1 m_{W_{R}}$
          \item Two isolated objects: one lepton and one AK8 jet.
          \item Signal in di-object mass spectrum $m_{\ell J}$
        \end{itemize}
      }
    \end{column}
  \end{columns}
\end{frame}

\note[itemize]{
  \footnotesize
  \item So when one of our final state quarks hadronizes, it produces a shower of particles that 
    pass through the CMS detector in a narrow cone. Rather than trying to track the energy and 
    momentum of each one of those particles, CMS clusters these particles together using something 
    called the anti kt algorithm and calls the collection of those particles a jet and we look at 
    the energy and momenta of that object as a whole.
  \item The sizes of these jets are defined in terms of the angular separation or angular distance 
    delta R, which is a measure of how collimated the particles in a jet are. The two most common 
    jet sizes are delta R = 0.4 and delta R = 0.8.
  \item There are two regions in the analysis, and so far I have only discussed this first one, the 
    resolved region. In the resolved search, we are searching for four isolated objects, either two 
    electrons or two muons, and two ak4 jets,. Each pair of objects have to be well separated, meaning 
    each pair has a delta R greater than 0.4 between them. And we search for signal in the four object 
    invariant mass spectrum.
  \item But, there is another analysis region called the boosted region. But, if the mass of the 
    right-handed W is much greater than the mass the heavy neutrino, then when the WR decays the heavy 
    neutrino will be produced with a large momentum, and then the three decay predicts of the heavy 
    neutrino will be highly collimated as they pass through the detector. So then rather than try to 
    reconstruct each of these objects individually, we cluster them into an AK8 jet, which we also call 
    a fat jet.
  \item In the boosted region, we have two objects: a well isolated lepton which is produced roughly 
    back-to-back with the heavy neutrino, which we reconstruct as a fat jet. We then use the invariant 
    mass of these two objects to search for signal.
  \item For the rest of this analysis, I am only going to focus on the resolved region. 
}

\begin{frame}{Analysis Backgrounds}
  \begin{block}{Dominant Backgrounds}
    The two main backgrounds that result in an $\ell \ell j j$ final state are 
    \boldcol{UMNMaroon}{Drell-Yan (DY)} and \boldcol{UMNMaroon}{$\mathbf{t\bar{t}}$}.
  \end{block}
  \vfill
  \begin{columns}
    \begin{column}{0.5\textwidth}
      \begin{figure}
        \centering
        \input{../figures/theory/dy-process}
        \caption{The Drell-Yan (DY) process.}
      \end{figure}
    \end{column}
    \begin{column}{0.5\textwidth}
      \begin{figure}
        \centering
        \input{../figures/theory/tt-process}
        \caption{$\mathrm{t\bar{t}}$ decay.}
      \end{figure}
    \end{column}
  \end{columns}
  \centering
  \boldcol{UMNSunny}{Minor Backgrounds include W+Jets, Single Top, Diboson, Triboson}
\end{frame}

\note[itemize]{
  \footnotesize
  \item The event selection targets our signal, but there are standard model processes that have 
    similar final states. The two primary backgrounds are Drell-Yan and ttbar.
  \item In Drell-Yan on the left here, two quarks annihilate and produce two same flavor leptons, 
    and then we get two jets coming from initial state radiation or final state radiation.
  \item For ttbar events, if you follow the decay of the top, you see the top decays to a b and 
    a left-handed W, the W then decays to a lepton and a neutrino. The antitop decays in a similar 
    fashion, and we are left with two leptons and two b quarks which hadronize to jets.
  \item And there are more minor background like single-top, W+jets, diboson and triboson.
  }

\subsection{Analysis Regions}

\begin{frame}{Analysis Regions}
  \begin{figure}
    \centering
    \begin{tikzpicture}[scale=5]
  
    % Define the division points for the three regions
    \def\xone{0.45}   % Right boundary of region 1 (SR)
    \def\xtwo{1.05}   % Right boundary of region 2 (CR) / left boundary of region 3 (SB)
    \def\xmax{2}      % Overall width of the box
    
    % Draw and fill the three vertical regions
    \fill [red!20!white]      % Region 1: SR
      (0,0) rectangle (\xone,1);
    \fill [blue!20!white]     % Region 2: CR (Control or Central Region)
      (\xone,0) rectangle (\xtwo,1);
    \fill [green!20!white]    % Region 3: SB
      (\xtwo,0) rectangle (\xmax,1);
      
    % Draw dashed vertical division lines for the three regions
    \draw[dashed,thick] (\xone,0) -- (\xone,1);
    \draw[dashed,thick] (\xtwo,0) -- (\xtwo,1);
    
    % Draw a horizontal dashed line at y = 0.5 (dividing OS and SS regions)
    \draw[dashed,thick] (-0.1,0.5) -- (\xmax,0.5);
    
    % Label the y-axis regions (lepton charges)
    \draw
      (-0.1,0.75) node[anchor=east] {SF}
      (0,0.65) node[anchor=east] {($ee$, $\mu\mu$)}
      (-0.1,0.25) node[anchor=east] {OF}
      (-0.08,0.15) node[anchor=east] {($e\mu$)};
    
    % Label the x-axis regions (lepton isolations)
    \draw
      (\xone,0) node[anchor=north] {150}
      (\xtwo,0) node[anchor=north] {400};
    
    % Add axis descriptions (the Lepton Flavor label remains on the left)
    \draw
      (-0.35,0.5) node[rotate=90,anchor=south] {Lepton Flavor};
    
    % Draw the x-axis as an arrow along the bottom edge.
    \draw[->, thick] (0,0) -- (\xmax+0.2,0) node[anchor=north] {$m_{ll}$ (GeV)};
    \draw[thick] (0,0) -- (0,1) node[anchor=west] {};

    % Add internal region labels (adjust positions as needed)
    \draw
      (\xone/2,0.78) node {DYCR}
      ({\xtwo + (\xmax-\xtwo)/2},0.25) node[align=center] {$e\mu$ Sideband}
      ({\xtwo + (\xmax-\xtwo)/2},0.75) node[align=center] {Signal Region};
    
\end{tikzpicture}
    \caption{A schematic diagram of the analysis region. The DY background is estimated from 
    the DY CR (blue).The backgrounds from $tW$ and $t\bar{t}$ production are estimated from the flavor CR 
    (green), where opposite flavor (OF) leptons are require.}
    \label{fig:dm-mass-scale}
  \end{figure}
\end{frame}

\note[itemize]{
  \footnotesize
  \item We define these different regions of phase space which we use to help estimate the backgrounds. 
  \item We estimate the backgrounds from Monte Carlo, but we use these different control regions in order to derive 
  corrections to the Monte Carlo in order to better predict the background in our Signal Region.
  \item We can see the Signal Region in the top right. The x-axis is the invariant mass of the two leptons in the final state. 
  So we require that the invariant mass of the two leptons by considerably high in order to reduce the background. 
  \item Moving to the left on the x-axis, you can see that if we lower that dilepton invariant mass requirement, 
  we get closer to the Z peak at 91.2 GeV. So we define a control region here, the Drell-Yan control region, right around 
  the Z peak because this is dominated by Drell-Yan and so we are able to drive corrections to our Drell-Yan Monte Carlo in 
  this control region. 
  \item Going back to our signal region and then moving down the y-axis, we see that in the signal region we require the leptons 
  to have the same flavor. But for ttbar events, the two tops are decaying independently, so they either can decay to the same flavor 
  or opposite flavor. So, when we require that our leptons be opposite flavor, we are in a region that is dominated by ttbar, and has 
  no signal within it either. So we are able to use this region to make more accurate corrections to our ttbar.
}

\section{Background Estimation}

\begin{frame}{Signal Region Backgrounds}
  \begin{columns}
    \begin{column}{0.50\textwidth}
      \centering
      \includegraphics[width=\textwidth]{../figures/plots/sr-bkg-frac-mumu.png}
    \end{column}
    \begin{column}{0.50\textwidth}
        \vspace*{-15mm}
        \centering
        \resizebox{\columnwidth}{!}{%
        \input{../figures/theory/signal-region}%
        }
      \vspace{1ex} 
      \begin{block}{Background Composition}
        As $m_{\ell\ell jj}$ increases, the background becomes more dominated by Drell-Yan (Z+jets).
      \end{block}
    \end{column}
  \end{columns}
\end{frame}

\subsection{$t\bar{t}$ and DY Modeling}

\begin{frame}{Flavor Control Region}
  \begin{columns}
    \begin{column}{0.50\textwidth}
      \centering
      \includegraphics[width=\textwidth]{../figures/plots/mass-fourobject-flavorcr.pdf}
    \end{column}
    \begin{column}{0.50\textwidth}
        \vspace*{-15mm}
        \centering
        \resizebox{\columnwidth}{!}{%
        \input{../figures/theory/flavor-cr}%
        }
      \vspace{1ex} 
      \begin{block}{$t\bar{t}$ Modeling}
        Simulation of $t\bar{t}$ agrees very well with data in the $e\mu$ sideband.
      \end{block}
    \end{column}
  \end{columns}
\end{frame}

\begin{frame}{Drell-Yan Control Region}
  \begin{block}{Mixed Results}
    DY modeling in the DY CR is mixed - better with leptonic variables and worse with hadronic variables.
  \end{block}
  \vfill
  \begin{columns}
    \begin{column}{0.35\textwidth}
      \includegraphics[width=\textwidth]{%
        ../figures/plots/pt-leadlep-dycr-mumu.pdf}
    \end{column}
    \begin{column}{0.35\textwidth}
      \includegraphics[width=\textwidth]{%
        ../figures/plots/pt-leadjet-dycr-mumu.pdf}
    \end{column}
    \begin{column}{0.3\textwidth}
 %     \vspace*{-15mm}
      \centering
      \resizebox{\columnwidth}{!}{%
      \input{../figures/theory/dy-cr}%
      }
    \end{column}
  \end{columns}
\end{frame}

\note[itemize]{
  \footnotesize
  \item For the rest of this talk I am going to discuss the Drell-Yan background estimation, which is the weakest aspect of this 
    analysis and it is what we are currently working on right now.
  \item As I mentioned before, the drell-yan background is simulated with Monte Carlo, and there several corrections 
    that we make to it in the control region. 
  \item We first apply QCD and nNLO electroweak corrections. We then notice that the jet pt spectrum is not well modeled when we 
    compare our Monte Carlo to data, so we then derived a jet pt correction in order to force the monte carlo to better model the 
    data. And then finally, we correct the normalization of the Drell-Yan monte carlo. These are a lot of corrections, the only 
    one that I am going to discuss is this third one, which is the jet pt reweighting. 
  \item When we first compared our Monte Carlo to data in the control region, we noticed that the leptonic side of the event agreed 
    well, so variables like the pt of the leading and subleading leptons were in good agreement, but the hadronic side of the event, 
    and specifically the leading jet pT spectrum, was mismodelled. 
  \item In particular, here we can see that the simulation is predicting harder jets than the data. 
  \item We think that this is due to a restriction in the Drell-Yan Monte Carlo event generator, which is restricted to the leading 
    order simulation of events and doesn’t take into account any loop corrections. Next-to-leading order calculations might improve 
    the accuracy of the simulation as the hard scattering would be calculated to one order higher in the strong coupling constant.
}

\subsection{Correcting the Background}

\begin{frame}{DY MC Reweighting}
  \begin{columns}
    \begin{column}{0.50\textwidth}
      \centering
      \includegraphics[width=\textwidth]{../figures/plots/mass_sf_ratio.pdf}
    \end{column}
    \begin{column}{0.50\textwidth}
        \centering
        \resizebox{0.55\columnwidth}{!}{%
        \input{../figures/theory/dy-cr}
        }
        \vfill
      \begin{block}{}
        \begin{enumerate}
          \item \boldcol{UMNMaroon}{Derive reweighting factors from the dijet mass $m_{jj}$.}
          \begin{itemize}
            \item Scale the DY MC in each bin to agree with data. 
          \end{itemize}
          \item Apply these scale factors to all the DY MC.
          \item Compare Data/MC in the Signal Region.
        \end{enumerate}
        
      \end{block}
    \end{column}
  \end{columns}
\end{frame}

\note[itemize]{
  \footnotesize
  \item The jet pT correction is a bin-by-bin rescaling given by this formula and apply this to our Drell Yan distributions in all 
    regions. Since the mismodelling is due to the hadronic side of the event, it doesn’t matter whether the leptons from the Z decay 
    in our events are electrons or muons, and so we take a weighted average of the two event samples, with our nominal ratio ranging 
    from 0.8 to 1.2. 
  \item Physcially what this means is that we are correcting the shape of the Drell-Yan leading jet pt distribution to match those 
    observed with data. 
}

\begin{frame}{DY CR Reweighted}
  \begin{columns}
    \begin{column}{0.69\textwidth}
      \begin{block}{}
        \centering
        The reweighting improves the Z+Jets modeling in the DY control region.
      \end{block}
    \end{column}
    \begin{column}{0.3\textwidth}
      \centering
      \resizebox{0.9\columnwidth}{!}{%
      \input{../figures/theory/dy-cr}
      }
    \end{column}
  \end{columns}
  \begin{columns}
    \begin{column}{0.35\textwidth}
      \begin{figure}
        \includegraphics[width=\textwidth]{%
        ../figures/plots/mlljj-dycr-mumu-pre.pdf}
        \caption{pre-ratio}
      \end{figure}
    \end{column}
    \begin{column}{0.35\textwidth}
      \begin{figure}
        \includegraphics[width=\textwidth]{%
        ../figures/plots/mlljj-dycr-mumu-post.pdf}
        \caption{post-ratio}
      \end{figure}
    \end{column}
    \begin{column}{0.3\textwidth}
      \begin{block}{}
        How does reweighting affect the signal region?
          \begin{itemize}
            \footnotesize
            \item Requires unblinding.
          \end{itemize}
      \end{block}
    \end{column}
  \end{columns}
\end{frame}

\note[itemize]{
  \footnotesize
  \item Here is a plot of our four object invariant mass in the Control Region on the left, and the signal region on the right. 
    This is the same control region plot that I showed a few slides ago, but again the blue line represents our background without 
    the Drell-Yan leading jet pT correction, and then the yellow Z+Jets background is the background with the correction applied. 
  \item So, we can see that in the control region, the correction pulls down the excess background and brings the simulation into 
    good agreement with data. However, in the signal region, we observe the opposite effect. Initially, there is good agreement 
    between simulation and data without the correction, and then when we apply the correction, we create a discrepancy, which is 
    especially noticeable above 2 TeV. 
}

\section{Looking Forward to Run 3}
\subsection{Previous Results}

\begin{frame}{Signal Region}
  \begin{columns}
    \begin{column}{0.50\textwidth}
      \centering
      \includegraphics[width=\textwidth]{../figures/plots/mlljj-sr-mumu.pdf}
    \end{column}
    \begin{column}{0.50\textwidth}
        \centering
        \resizebox{0.55\columnwidth}{!}{%
        \input{../figures/theory/signal-region.tex}
        }

      \vfill
      \boldcol{UMNMaroon}{The lead jet $p_{T}$ has been reweighted in previous studies.}
      
      \begin{block}{Reweighting with the lead jet $p_T$}
        \begin{itemize}
          \item Before the correction, there is good agreement.
          \item The corrections pulls the MC down, out of agreement.
          \item Excess data isn't very peak-like.
        \end{itemize}
      \end{block}
    \end{column}
  \end{columns}
\end{frame}

\subsection{Optimization Studies}

\begin{frame}{$m_{\ell \ell}$ optimization}
  \begin{columns}
    \begin{column}{0.50\textwidth}
      \centering
      \includegraphics[width=\textwidth]{../figures/plots/mll-optimization.png}
    \end{column}
    \begin{column}{0.50\textwidth}
 %       \vspace*{-5mm}
        \centering
        \resizebox{\columnwidth}{!}{%
        \input{../figures/theory/signal-region.tex}%
        }
 %     \vspace{1ex} 
      \begin{block}{Defining the signal region}
        Is $m_{\ell \ell} > 400 \mathrm{~GeV}$ the best cut for the signal region?
        \begin{itemize}
          \item Loosening lets in more DY background.
          \item Tightening limits statistical power.
        \end{itemize}
      \end{block}
    \end{column}
  \end{columns}
\end{frame}

\section{Conclusion} % \ssection{Conclusion}

\begin{frame}{Conclusion}
  Many improvements and areas of study under development in the $\mathrm{W_R}$ and $\mathrm{N}$ search:

  \begin{columns} % [t] for top alignment of columns
    %――――――――――――――――――――――――――――
    \begin{column}{0.5\textwidth}
      \begin{block}{Looking Forward to Run 3}
        Beginning comparisons for Run 3 Data/MC.
      \end{block}

      \begin{block}{Futher Optimization Studies}
        \begin{itemize}
          \item Looking at how \boldcol{UMNMaroon}{adding a third jet} to the final‐state objects affects 
          signal and background yields.
        \end{itemize}
      
      \end{block}
    \end{column}

    %――――――――――――――――――――――――――――
    \begin{column}{0.5\textwidth}
      \centering
      \includegraphics[width=0.75\textwidth]{wr-boson-chatgpt.png}
    \end{column}
  \end{columns}
\end{frame}

\begin{backup}

\begin{frame}{LHC}
    \begin{figure}
      \centering
      \includegraphics[width=0.7\textwidth]{../figures/experiment/lhc-schematic.png}
      \caption{Schematic showing the overall layout of the LHC. Image credit -- CERN.}
    \end{figure}
\end{frame}

\end{backup}

\end{document} 
