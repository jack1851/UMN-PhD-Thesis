\input{../preamble}

%--------------------------------------------------
% Geometry and Scale Settings
% Adjust the scaling factor for physical dimensions.
%--------------------------------------------------
\geometry{mag=1600, truedimen}

%--------------------------------------------------
% University of Minnesota Color Definitions
%--------------------------------------------------
\definecolor{UMNMaroon}{RGB}{122,0,25}
\definecolor{UMNLightGold}{RGB}{255,215,95}
\definecolor{UMNGold}{RGB}{255,204,51}
\definecolor{UMNStormy}{RGB}{64,77,91}
\definecolor{UMNSunny}{RGB}{0,149,182}
\definecolor{UMNLightGray}{RGB}{255,222,122}

%--------------------------------------------------
% Package Imports
% Load additional packages required for the presentation.
%--------------------------------------------------
\usepackage{xspace}         % Ensures \xspace works properly with custom macros.
\usepackage{textpos}        % Provides absolute positioning.
\usepackage{epstopdf}       % Converts EPS images to PDF.
\usepackage{graphicx}       % Handles image inclusion.
\usepackage{beamerthemesplit} % Legacy beamer theme; consider using a modern theme if desired.
\usepackage{soul}           % Allows text highlighting and underlining.
\usepackage{xifthen}        % Enables conditional statements.
\usepackage{listings}       % For code formatting.
\usepackage{hyperref}       % For hyperlink support in the generated PDF.
\usepackage{bbding}         % Provides additional symbols (e.g., checkmarks).

%--------------------------------------------------
% Hyperref Configuration
% Customize link colors and PDF metadata.
%--------------------------------------------------
\hypersetup{
    colorlinks=true,
    linkcolor=UMNMaroon,
    urlcolor=UMNStormy,
    citecolor=UMNSunny,
    pdftitle={Presentation Title},
    pdfauthor={Your Name}
}

%--------------------------------------------------
% Listings Settings for Code Snippets
%--------------------------------------------------
\definecolor{backcolour}{rgb}{0.95,0.95,0.92}
\lstdefinestyle{mystyle}{
    backgroundcolor=\color{UMNLightGray},
    commentstyle=\color{UMNStormy},
    keywordstyle=\color{UMNMaroon},
    numberstyle=\tiny\color{UMNMaroon},
    stringstyle=\color{UMNSunny},
    basicstyle=\ttfamily\footnotesize,
    emph={ldmx },
    emphstyle=\color{UMNMaroon},
    breakatwhitespace=false,
    breaklines=true,
    captionpos=b,
    keepspaces=true,
    numbers=left,
    numbersep=5pt,
    showspaces=false,
    showstringspaces=false,
    showtabs=false,
    tabsize=2
}
\lstset{style=mystyle}

%--------------------------------------------------
% Custom Commands and Environments
%--------------------------------------------------

%--- Code Listing Commands ---%
% Import code from an external file.
% #1: Options for the formatter.
% #2: File path to the code file.
\newcommand{\codefile}[2]{%
    \lstinputlisting[#1]{#2}
}%

% Inline code command with a styled box.
% #1: Inline code text.
\newcommand{\code}[1]{%
  \tikz[baseline=(codebox.base)]{
    \node[inner sep=1pt,outer sep=0pt,draw=UMNStormy,line width=0.05mm,fill=UMNLightGray,rounded corners=0.03cm] (codebox) 
      {\textcolor{UMNSunny}{#1}};
  }%
}%

% Styled title command within a frame for subsection headings.
% #1: Text to style.
\newcommand{\framesection}[1]{%
  \boldcol{UMNStormy}{#1}\hspace{1em}%
}%

%--- List Item Symbols ---%
% Define checkmark and crossmark symbols with custom colors.
\def\checkmark{{\color{UMNSunny}\Checkmark}}
\def\crossmark{{\color{UMNMaroon}\XSolidBrush}}

% Define "done" and "todo" list items.
\def\done{\item[$\color{UMNMaroon}\boxtimes$]}
\def\todo{\item[$\color{UMNMaroon}\square$]}

% Define "good" and "bad" list items.
\def\good{\item[\checkmark]}
\def\bad{\item[\crossmark]}

%--- Hyperlink Button ---%
% Creates a clickable button linking to a URL.
% #1: URL to link to.
% #2: Button text.
\newcommand{\hlink}[2]{%
    \href{#1}{\beamergotobutton{#2}}
}%

%--- Bold and Colored Text ---%
% Boldens and colors text.
% #1: Color.
% #2: Text.
\newcommand{\boldcol}[2]{%
    {\color{#1}\textbf{#2}}\xspace
}%

%--- One Plot Slide ---%
% Creates a slide with one plot that fills most of the frame.
% #1: Slide Title.
% #2: Slide Subtitle.
% #3: File path to the image.
\newcommand{\oneplotslide}[3]{%
    \begin{frame}
        \frametitle{#1}
        \framesubtitle{#2}
        \begin{figure}
            \centering
            \includegraphics[width=\textwidth,height=0.8\textheight,keepaspectratio]{#3}
        \end{figure}
     \end{frame}
}%

%--- Four Tiles Slide ---%
% Arranges content into four tiles on a slide.
% #1: Slide Title.
% #2: Slide Subtitle.
% #3: Top left content.
% #4: Top right content.
% #5: Bottom left content.
% #6: Bottom right content.
\newcommand{\fourtileslide}[6]{%
    \begin{frame}
        \frametitle{#1}
        \framesubtitle{#2}
        \begin{figure}[h]
            \centering
            \begin{tabular}{cc}
                #3 & #4 \\
                #5 & #6
            \end{tabular}
        \end{figure}
    \end{frame}
}%

%--- Four Plots Slide ---%
% A slide that tiles four plots; empty paths yield a blank tile.
% #1: Slide Title.
% #2: Slide Subtitle.
% #3: Top left image file path.
% #4: Top right image file path.
% #5: Bottom left image file path.
% #6: Bottom right image file path.
\newcommand{\fourplotslide}[6]{%
    \fourtileslide{#1}{#2}%
        {\ifthenelse{\isempty{#3}}{}{\includegraphics[height=0.4\textheight]{{#3}}}}%
        {\ifthenelse{\isempty{#4}}{}{\includegraphics[height=0.4\textheight]{{#4}}}}%
        {\ifthenelse{\isempty{#5}}{}{\includegraphics[height=0.4\textheight]{{#5}}}}%
        {\ifthenelse{\isempty{#6}}{}{\includegraphics[height=0.4\textheight]{{#6}}}}%
}%

%--- Figure with Comments ---%
% Creates a slide with an image on the left and commentary on the right.
% #1: File path of the image.
\newenvironment{figurecomments}[1]{%
  \begin{columns}
    \begin{column}{0.6\textwidth}
      \begin{figure}
        \centering
        \includegraphics[height=0.7\textheight]{#1}
      \end{figure}
    \end{column}
    \begin{column}{0.4\textwidth}
      \footnotesize
}{%
    \end{column}
  \end{columns}
}%

%--- Section Title Slide ---%
% Creates a full-slide title card for sections.
% #1: Section title.
\newcommand{\sectionframe}[1]{%
\begin{frame}
    \vfill
    \centering
    \begin{beamercolorbox}[sep=8pt,center,shadow=true,rounded=true]{title}
        \usebeamerfont{title}#1\par%
    \end{beamercolorbox}
    \vfill
\end{frame}
}%

%--- Backup Slides Environment ---%
% Wrap backup slides to exclude them from the main slide count.
% Optionally, you can title the backup section (default is "Questions").
\newenvironment{backup}[1][Questions]{%
  \sectionframe{#1}
  \newcounter{finalframe}
  \setcounter{finalframe}{\value{framenumber}}
}{%
  \setcounter{framenumber}{\value{finalframe}}
}%