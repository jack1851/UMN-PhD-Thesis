%---------------------------------------------------------------------
% UMN Custom Beamer Theme Preamble
%---------------------------------------------------------------------
\input{../preamble} % External preamble for general packages
% Note: Ensure that TikZ, xspace and other dependencies are loaded here if not loaded below

%---------------------------------------------------------------------
% Geometry Adjustments
%---------------------------------------------------------------------
\geometry{mag=1600,truedimen} % Adjust layout scaling

%---------------------------------------------------------------------
% UMN Color Definitions (University of Minnesota colors)
%---------------------------------------------------------------------
\definecolor{UMNMaroon}{RGB}{122,0,25}
\definecolor{UMNLightGold}{RGB}{255,215,95}
\definecolor{UMNGold}{RGB}{255,204,51}
\definecolor{UMNStormy}{RGB}{64,77,91}
\definecolor{UMNSunny}{RGB}{0,149,182}
\definecolor{UMNLightGray}{RGB}{213,214,210}

\definecolor{GopherMaroon}{RGB}{122, 0, 25}
\definecolor{GopherGold}{RGB}{255, 204, 51}
\definecolor{GopherLightGold}{RGB}{255, 222, 122}
\definecolor{GopherDarkMaroon}{RGB}{91, 0, 19}

%---------------------------------------------------------------------
% Packages specific to this theme
%---------------------------------------------------------------------
\usepackage{textpos}    % For precise positioning
\usepackage{epstopdf}    % Convert EPS images to PDF
\usepackage{soul}        % For highlighting and letter spacing
\usepackage{xifthen}     % For conditional logic in commands
\usepackage{listings}    % For code listings


\usetikzlibrary{calc,intersections}   
% Define a custom background color for listings if needed
%\definecolor{backcolour}{rgb}{0.95,0.95,0.92} % Consider using this for consistency

%---------------------------------------------------------------------
% Listing Style Definition
%---------------------------------------------------------------------
\lstdefinestyle{mystyle}{
    backgroundcolor=\color{UMNLightGray},
    commentstyle=\color{UMNStormy},
    keywordstyle=\color{UMNMaroon},
    numberstyle=\tiny\color{UMNMaroon},
    stringstyle=\color{UMNSunny},
    basicstyle=\ttfamily\footnotesize,
    emph={ldmx}, % Note: Ensure no trailing spaces unless intentional
    emphstyle=\color{UMNMaroon},
    breakatwhitespace=false,
    breaklines=true,
    captionpos=b,
    keepspaces=true,
    numbers=left,
    numbersep=5pt,
    showspaces=false,
    showstringspaces=false,
    showtabs=false,
    tabsize=2
}
\lstset{style=mystyle}

%---------------------------------------------------------------------
% Custom Commands and Environments
%---------------------------------------------------------------------

% Bold and color text command.
\newcommand{\boldcol}[2]{%
    {\color{#1}\textbf{#2}}\xspace
}%

% Slide section title command for intra-slide division.
\newcommand{\framesection}[1]{%
  \boldcol{UMNStormy}{#1}\hspace{1em}%
}%

% Code inclusion commands for listings.
\newcommand{\codefile}[2]{%
    \lstinputlisting[#1]{#2}
}%

\newcommand{\code}[1]{%
  \tikz[baseline=(codebox.base)]{
    \node[inner sep=1pt,outer sep=0pt,draw=UMNStormy,line width=0.05mm,fill=UMNLightGray,rounded corners=0.03cm] (codebox) 
      {\textcolor{UMNSunny}{#1}};
  }%
}%

% Checkmark and crossmark definitions using bbding.
\usepackage{bbding}
\newcommand{\checkmarkicon}{{\color{UMNSunny}\Checkmark}}
\newcommand{\crossmarkicon}{{\color{UMNMaroon}\XSolidBrush}}

% List item markers for done, todo, good, and bad items.
\newcommand{\done}{\item[$\color{UMNMaroon}\boxtimes$]}
\newcommand{\todo}{\item[$\color{UMNMaroon}\square$]}
\newcommand{\good}{\item[\checkmarkicon]}
\newcommand{\bad}{\item[\crossmarkicon]}

% Hyperlink button command
\newcommand{\hlink}[2]{%
    \href{#1}{\beamergotobutton{#2}}
}%

% Slide layout commands for including plots/images.
\newcommand{\oneplotslide}[3]{% #1=Title, #2=Subtitle, #3=Image path
    \begin{frame}
        \frametitle{#1}
        \framesubtitle{#2}
        \begin{figure}
            \centering
            \includegraphics[width=\textwidth,height=0.8\textheight,keepaspectratio]{#3}
        \end{figure}
    \end{frame}
}%

\newcommand{\fourtileslide}[6]{% #1=Title, #2=Subtitle, #3-#6=Tile contents
    \begin{frame}
        \frametitle{#1}
        \framesubtitle{#2}
        \begin{figure}[h]
            \centering
            \begin{tabular}{cc}
                #3 & #4 \\
                #5 & #6
            \end{tabular}
        \end{figure}
    \end{frame}
}%

\newcommand{\fourplotslide}[6]{%
    \fourtileslide{#1}{#2}%
        {\ifthenelse{\isempty{#3}}{}{ \includegraphics[height=0.4\textheight]{#3}}}%
        {\ifthenelse{\isempty{#4}}{}{ \includegraphics[height=0.4\textheight]{#4}}}%
        {\ifthenelse{\isempty{#5}}{}{ \includegraphics[height=0.4\textheight]{#5}}}%
        {\ifthenelse{\isempty{#6}}{}{ \includegraphics[height=0.4\textheight]{#6}}}%
}%

% Environment for a slide with a picture and accompanying comments.
\newenvironment{figurecomments}[1]{% #1=Image path
  \begin{columns}
    \begin{column}{0.6\textwidth}
      \begin{figure}
        \centering
        \includegraphics[height=0.7\textheight]{#1}
      \end{figure}
    \end{column}
    \begin{column}{0.4\textwidth}
      \footnotesize
}{%
    \end{column}
  \end{columns}
}%

% A full slide for section titles.
\newcommand{\sectionframe}[1]{%
  \begin{frame}
    \vfill
    \centering
    \begin{beamercolorbox}[sep=8pt,center,shadow=true,rounded=true]{title}
        \usebeamerfont{title}#1\par%
    \end{beamercolorbox}
    \vfill
  \end{frame}
}%

% Backup slides environment to exclude from main slide count.
\newenvironment{backup}[1][Questions]{%
  \sectionframe{#1}%
  \newcounter{finalframe}%
  \setcounter{finalframe}{\value{framenumber}}%
}{%
  \setcounter{framenumber}{\value{finalframe}}%
}%

%---------------------------------------------------------------------
% Beamer Presentation Mode Settings
%---------------------------------------------------------------------
\mode<presentation> {
  \usetheme{Madrid}
  \usefonttheme{serif}
  \addtobeamertemplate{frametitle}{}{%
      \begin{textblock*}{100mm}(0.85\textwidth,-0.95cm)
          \includegraphics[height=0.9cm]{goldy_leaning.png}
      \end{textblock*}
      \begin{textblock*}{100mm}(0.95\textwidth,-0.95cm)
        \includegraphics[height=0.9cm]{cms_logo.png}
      \end{textblock*}
  }

  % Beamer color definitions and palette
  \setbeamercolor{structure}{fg=UMNStormy} % Colors blocks and the 'Figure' caption heading. Default fg=UMNStormy.
  \setbeamercolor{palette primary}{fg=UMNMaroon, bg=GopherLightGold} % Bottom right corner. fg is text, bg is backgroun. Default fg=UMNMaroon, bg=UMNLightGray.
  \setbeamercolor{palette secondary}{fg=UMNMaroon, bg=GopherGold} % Bottom middle. Default fg=UMNMaroon, bg=white.
  \setbeamercolor{palette tertiary}{fg=UMNMaroon, bg=GopherLightGold} % Bottom left. Default fg=UMNLightGold, bg=UMNStormy
  \setbeamercolor{frametitle}{fg=UMNMaroon, bg=GopherLightGold} % Title of each frame. Default fg=UMNLightGold, bg=UMNMaroon
  \setbeamercolor{title}{fg=UMNMaroon, bg=GopherLightGold} % First title slide. Default fg=UMNMaroon, bg=UMNLightGray
  \setbeamercolor{section in toc}{fg=UMNMaroon} % Table of content title colors. Default fg=UMNMaroon
  \setbeamercolor{section in toc shaded}{fg=UMNMaroon} % ? Default fg=UMNMaroon
  \setbeamercolor{button}{fg=UMNLightGold, bg=UMNMaroon} % ? Default fg=UMNLightGold, bg=UMNMaroon
  \setbeamercolor{palette sidebar secondary}{fg=UMNMaroon} % ? Default fg=UMNLightGold, bg=UMNMaroon
  \setbeamercolor{section in sidebar shaded}{fg=UMNMaroon} % ? Default fg=UMNLightGold, bg=UMNMaroon

  \setbeamertemplate{itemize item}{\color{UMNMaroon}$\blacksquare$}
  \setbeamertemplate{itemize subitem}{\color{UMNLightGold}$\blacktriangleright$}
  \setbeamertemplate{enumerate items}[default]
  \setbeamertemplate{sections/subsections in toc}[sections numbered]

  % Remove navigation symbols if not needed
  \setbeamertemplate{navigation symbols}{}

  \setbeamercolor{block body alerted}{fg=UMNSunny, bg=UMNMaroon!20}
  \setbeamercolor{block title alerted}{fg=UMNLightGold, bg=UMNMaroon}
}

%---------------------------------------------------------------------
% Author and Institution Details
%---------------------------------------------------------------------
\newcommand{\with}{} % Placeholder; fill or remove if not needed

\author{Billy Jackson} % Your name
\institute[UMN]{%
  UMN HEP Seminar \\[2mm]
  University of Minnesota \\[2mm]
  \href{mailto:jack1851@umn.edu}{jack1851@umn.edu} \\[2mm]
  \begin{tabular}{p{0.4\textwidth}}\centering\with\end{tabular}
}